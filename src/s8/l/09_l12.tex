%
% This file is part of Calicut University Question Paper Collection.
%
% Copyright (c) 2012-2015 Mohammed Sadik P. K. <sadiq (at) sadiqpk (d0t) org>.
% License: GNU GPLv3 or later
%
% Calicut University Question Paper Collection is free software: you can
% redistribute it and/or modify
% it under the terms of the GNU General Public License as published by
% the Free Software Foundation, either version 3 of the License, or
% (at your option) any later version.
% 
% Calicut University Question Paper Collection is distributed in the hope
% that it will be useful,
% but WITHOUT ANY WARRANTY; without even the implied warranty of
% MERCHANTABILITY or FITNESS FOR A PARTICULAR PURPOSE.  See the
% GNU General Public License for more details.
% 
% You should have received a copy of the GNU General Public License
% along with Calicut University Question Paper Collection.
% If not, see <http://www.gnu.org/licenses/>.
% 
%
\def \subj{AI 09 804 L12---SOFT COMPUTING TECHNIQUES}

\mainhead{C 80513}{2}
\semeight{APRIL 2015}
\sub{\subj}
\maxtime

\partA

\iitem What is artificial intelligence?
\item What is an activation function?
\item Define momentum factor.
\item What is fuzzification.
\item What is the significance of fitness function in Genetic algorithms?

\markA
\partB

\item Explain the model of a Perceptron.
\item What is MADALINE? Give its structure and output function.
\item Explain weight update concepts of Levenberg Marquardt and resilient
  BP algorithms.
\item Explain Fuzzy If-Then rule with an example.
\item Explain the Fuzzy Inference System.
\item Explain the application of Genetic algorithms.

\markB
\partC

\item \iitem With suitable example explain the architecture of an Artificial Neural
  Network.
\Or
\item Describe McCulloch and Pitts Model.
\ene

\newpage \again

\item \iitem With suitable example explain the input layer, hidden layer and output layer
  of Back Propagation Neural Network.
\Or
\item \iitem What is the effect of learning rate in a Back Propagation Neural Network?
\item Explain Local Minima problem in Back Propagation Neural Network.
\ene
\ene

\item \iitem Explain Fuzzy sets and relations.
\Or
\item With a suitable case study explain the design of a Fuzzy Logic Controller.
\ene

\item \iitem Discuss the basic concepts of Genetic algorithm.
\Or
\item Explain the basic concepts of Genetic Programming.
\ene

\markC
\ene

\newpage

\mainhead{C 60505}{2}
\semeight{APRIL 2014}
\sub{\subj}
\maxtime

\partA

\iitem What is Soft computing?
\item Define Learning rate.
\item What is generalized modus ponens (GMP)?
\item What is defuzzification?
\item What are the advantages of Genetic algorithms?

\markA
\partB

\item Explain the model of an artificial neuron.
\item What is ADALINE? Give its structure and output function.
\item Explain why an MLP does not learn if the initial weights and
  biases are all zeros.
\item Compare Mamdani and Sugeno Fuzzy models.
\item Explain the application of extension principle to fuzzy sets with
  (i) discrete universes; and (ii) continuous universes.
\item Explain the common characteristics shared by derivative free optimization
  methods.

\markB
\partCo

\item \iitem Explain the various activation functions.
\Or

\newpage \again

\item \iitem Explain off-line learning and on-line learning.
\item Explain the adaptive network and its error-propagation model.
\ene
\ene

\item \iitem Explain the various training algorithms and compare their
  performances.
\Or
\item Derive the back propagation learning rule for a single-output RBFN,
  where the parameters include the center ($u_1$), the width ($\sigma_i$)
  and the connection weight ($c_i$) for each receptive field.
\ene

\item \iitem Prove Associativity, Distributivity over union, weak distributivity
  over intersection and monotonicity for max-min composition relation.
\Or
\item Derive the partial derivatives of a Gaussian Membership Function
  $y =$ Gaussian ($x; c, \sigma$) with respect to its argument $x$ and
  parameters $c$ and $\sigma$.
\ene

\item \iitem Explain the following terms of Genetic algorithms:
\iitem Fitness Evaluation.
\item Selection
\item Crossover
\item Mutation
\ene
\Or
\item Explain simulated annealing with the following terminologies:
\iitem Objective function
\item Generating function.
\item Acceptance function.
\item Annealing schedule.
\ene
\ene

\markC
\ene

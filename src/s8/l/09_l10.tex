%
% This file is part of Calicut University Question Paper Collection.
%
% Copyright (c) 2012-2015 Mohammed Sadik P. K. <sadiq (at) sadiqpk (d0t) org>.
% License: GNU GPLv3 or later
%
% Calicut University Question Paper Collection is free software: you can
% redistribute it and/or modify
% it under the terms of the GNU General Public License as published by
% the Free Software Foundation, either version 3 of the License, or
% (at your option) any later version.
% 
% Calicut University Question Paper Collection is distributed in the hope
% that it will be useful,
% but WITHOUT ANY WARRANTY; without even the implied warranty of
% MERCHANTABILITY or FITNESS FOR A PARTICULAR PURPOSE.  See the
% GNU General Public License for more details.
% 
% You should have received a copy of the GNU General Public License
% along with Calicut University Question Paper Collection.
% If not, see <http://www.gnu.org/licenses/>.
% 
%
\def \subj{AI 09 804 L10---ROBOTICS \& AUTOMATION}

\mainhead{C 80512}{2}
\semeight{APRIL 2015}
\sub{\subj}
\maxtime

\partA

\iitem Define Robotics.
\item What are the common types of arms of robot?
\item Write short note on actuators.
\item What is the function of manipulator?
\item What is forward kinematics?

\markA
\partB

\item Define a degree of freedom and discuss the various types of degree of
  freedom.
\item Compare electric and hydraulic drives.
\item Explain the working principle of acoustic sensor.
\item With the aid of neat diagram, explain the principle of robot manipulators.
\item Explain any two types of grippers with neat sketch.
\item Explain hill climbing robots.

\markB
\partC

\item \iitem Explain in detail about different types of generations of robots.
\Or
\item Explain dynamic stabilization of robots and mention the three laws of
  Asimov on robots.
\ene

\newpage \again

\item \iitem What are the advantages of using pneumatic drives in robots? Discuss
  the different types of pneumatic drives used in the robots with the help of neat
  sketches.
\Or
\item How do you determine the HP of a motor and gearing ratio for wint joint of a
  robot having 1 Kg. pay load.
\ene

\item \iitem Briefly explain various force control methods in robot manipulators.
\Or
\item How is a robot end effector specified? Discuss the design considerations in
  the robot end of the arm tooling.
\ene

\item \iitem Give the jacobian matrix for a cylindrical robot with 3 degrees of freedom.
\Or
\item What are the various inputs to an inverse kinematics
  algorithm? Explain functioning of an kinematics algorithm.
\ene

\markC
\ene

\newpage

\mainhead{41598}{2}
\semeight{APRIL 2013}
\sub{\subj}
\maxtime

\partA

\iitem Define robotics.
\item What are the common types of arms of robot?
\item Write short note on actuators.
\item What is the function of manipulator?
\item What is forward kinematics?

\markA
\partB

\item Define a degree of freedom and discuss the various types of degree of freedom.
\item Compare electric and hydraulic drives.
\item Explain the working principle of acoustic sensor.
\item With the aid of neat diagram, explain the principle of robot manipulators.
\item Explain any two types of grippers with neat sketch.
\item Explain hill climbing robots.

\markB
\partC

\item \iitem Explain in detail about different types of generations of robots.
\Or
\item Explain dynamic stabilization of robots and mention the three laws of Asimov
  on robots.
\ene

\newpage \again

\item \iitem What are the advantages of using pneumatic drives in robots? Discuss the
  different types of pneumatic drives used in the robots with the help of
  neat sketches.
\Or
\item How do you determine the HP of a motor and gearing ratio for wint joint 
  of a robot having 1 Kg pay load.
\ene

\item \iitem Briefly explain various force control methods in robot manipulators.
\Or
\item How is a robot and effector specified? Discuss the design considerations
  in the robot and the arm tooling.
\ene

\item \iitem Give the jacobian matrix for a cylindrical robot with 3 degrees of freedom.
\Or
\item What are the various inputs to an inverse kinematics
  algorithm? Explain functioning of an kinematics algorithm.
\ene

\markC
\ene

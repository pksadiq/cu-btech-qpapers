%
% This file is part of Calicut University Question Paper Collection.
%
% Copyright (c) 2012-2015 Mohammed Sadik P. K. <sadiq (at) sadiqpk (d0t) org>.
% License: GNU GPLv3 or later
%
% Calicut University Question Paper Collection is free software: you can
% redistribute it and/or modify
% it under the terms of the GNU General Public License as published by
% the Free Software Foundation, either version 3 of the License, or
% (at your option) any later version.
% 
% Calicut University Question Paper Collection is distributed in the hope
% that it will be useful,
% but WITHOUT ANY WARRANTY; without even the implied warranty of
% MERCHANTABILITY or FITNESS FOR A PARTICULAR PURPOSE.  See the
% GNU General Public License for more details.
% 
% You should have received a copy of the GNU General Public License
% along with Calicut University Question Paper Collection.
% If not, see <http://www.gnu.org/licenses/>.
% 
%
\def \subj{AI 09 802---DATA AND COMPUTER COMMUNICATIONS}

\mainhead{C 80508}{2}
\semeight{APRIL 2015}
\sub{\subj}
\maxtime

\partA

\iitem Name any \emph{two} guided media used for data communications.
\item What is Time Division Multiplexing?
\item Why hexagonal geometry is preferred for cellular mobile communications?
\item Compare circuit switching and packet switching.
\item What is internetworking?

\markA
\partB

\item What are Multi-Service networks? Explain.
\item Compare Synchronous and Asynchronous transmission.
\item With neat sketch explain digital subscriber line.
\item What is flow control and link management? Explain.
\item Explain high speed LANs.
\item Explain the general requirements for a protocol.

\markB
\partC

\item \iitem \iitem Explain the standard organizations for data communications.
\item Compare serial and parallel data transmission.
\ene
\Or

\newpage \again

\item \iitem Explain PSDN and broadband networks.
\item Explain channel capacity and its limitations.
\ene
\ene

\item \iitem What is the need of synchronization in data communication? Explain
  bit and character synchronization.
\Or
\item Briefly explain the various error detection and correction schemes used in
  data communications.
\ene

\item \iitem Explain LAN and its topologies.
\Or
\item Explain Asynchronous Transfer Mode.
\ene

\item \iitem Discuss in detail about Internet Architecture and protocol.
\Or
\item Explain in detail about Network security and its applications.
\ene

\markC
\ene

\newpage

\mainhead{C 60501}{2}
\semeight{APRIL 2014}
\sub{\subj}
\maxtime

\partA

\iitem State the advantage of digital transmission over analog transmission.
\item What are the three types of characters used in data communication codes?
\item State the difference between circuit switching and packet switching.
\item Name any \emph{two} IEEE LAN standards.
\item What is DHCP?

\markA
\partB

\item Explain the four modes of transmission for data communications.
\item Compare guided and wireless communication channels.
\item Explain the asynchronous and synchronous data formats used to achieve
  character synchronization in data communication.
\item Briefly describe any \emph{two} Ethernet schemes.
\item Explain the token and data frame structure of FDDI.
\item What are some of the possible services that a link-layer protocol can offer
  to the network layer? Which of these link-layer services have corresponding
  services in IP? In TCP?

\markB
\partC

\item \iitem With simplified block diagram, explain data communication network.

\newpage \again

\Or
\item Prove that the equivalent bandwidth of a source increases with its variance
  and decreases with the acceptable average delay through the queue.
\ene

\item \iitem For a 12-bit data string of 101100010010, determine the number of Hamming
  bits required, arbitrarily place the Hamming bits into the data string, determine the
  condition of each Hamming bit, assume an arbitrary single-bit transmission error,
  and prove that the Hamming code will detect the error.
\Or
\item Determine the Block Check Sequence for the following data and CRC generating
  polynomials:\\
  Data G($x$) $= x^7 + x^5 + x^4 + x^2 + x^1 + x^0$\\
  CRC P($x$) $=  x^5 + x^4 + x^1 + x^0$
\ene

\item \iitem Explain the various subfields used with ATM header field and Information
  field.
\Or
\item Explain Cellular Wireless Networks and its applications.
\ene

\item \iitem Discuss in detail about Internet Protocol.
\Or
\item Explain in detail about Network security.
\ene

\markC
\ene

\newpage

\mainhead{41595}{2}
\semeight{APRIL 2013}
\sub{\subj}
\maxtime

\partA

\iitem What are the different data transmission methods?
\item What is meant by channel capacity?
\item Explain CDMA.
\item Differentiate MESH and RING topology.
\item Explain the functions of i. SMTP protocol ii. FTP protocol.

\markA
\partB

\item Explain the concept of data transmission using multi-service networks.
\item Differentiate synchronous and asynchronous transmission technique.
\item Explain time division multiplexing.
\item Differentiate wireless LAN and wired LAN.
\item Explain congestion control in switched data network.
\item Explain Internet architecture.

\markB
\partC

\item \iitem Briefly explain different encoding techniques.
\Or
\item Explain different types of guided transmission media.
\ene

\newpage \again

\item \iitem Explain IEEE 802.11.
\Or
\item Explain different error detection methods.
\ene

\item \iitem Briefly explain different types of topologies in network.
\Or
\item Explain different switching methods.
\ene

\item \iitem  Explain different network security methods.
\Or
\item Briefly explain
\iitem Transport Protocol
\item Application support Protocol
\ene
\ene

\markC
\ene

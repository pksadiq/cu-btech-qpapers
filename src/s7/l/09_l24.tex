%
% This file is part of Calicut University Question Paper Collection.
%
% Copyright (c) 2012-2015 Mohammed Sadik P. K. <sadiq (at) sadiqpk (d0t) org>.
% License: GNU GPLv3 or later
%
% Calicut University Question Paper Collection is free software: you can
% redistribute it and/or modify
% it under the terms of the GNU General Public License as published by
% the Free Software Foundation, either version 3 of the License, or
% (at your option) any later version.
% 
% Calicut University Question Paper Collection is distributed in the hope
% that it will be useful,
% but WITHOUT ANY WARRANTY; without even the implied warranty of
% MERCHANTABILITY or FITNESS FOR A PARTICULAR PURPOSE.  See the
% GNU General Public License for more details.
% 
% You should have received a copy of the GNU General Public License
% along with Calicut University Question Paper Collection.
% If not, see <http://www.gnu.org/licenses/>.
% 
%
\def \subj{CS 09 L24---COMPUTER BASED NUMERICAL METHOD}


\mainhead{29691}{2}

\semseven{OCTOBER 2012}

\sub{\subj}

\maxtime


\partA

\iitem What is inherent error?

\item Define the rate of convergence of an iterative method.

\item What is the difference between Stirling and Bessel interpolation?

\item What is weight function as used in numerical integration?

\item Define a cubic spline.


\markA

\partB


\item Explain the Newton Raphson method.

\item Determine the appropriate step size to use in the construction of a table of $f$($x$) $ = (1 + x)^6$
  on [0, 1]. The truncation error for linear interpolation is to be bounded by $5\times 10^{-5}$.

\item What is the relation between Bessel's and Everett's formulae?

\item Explain the trapizoidal rule for numerical integration.

\item Explain the method of least square as a curve fitting procedure.

\item Explain the use of frequency chart.


\markB

\partCo


\item \iitem Discuss the propagation of error for the following:-

\iitem The sum of three numbers.

\[p + q + r = ( \bar{p} + \epsilon_p ) + ( \bar{q} + \epsilon_q) + (\bar{r} + \epsilon_r).\]

\item The quotient of three numbers.

\[ \frac{p}{q} = \frac{\bar{p} + \epsilon_p}{\bar{q} + \epsilon_q}.\]

\item The product of three numbers.

\newpage

\again

\ene
\ene
\ene

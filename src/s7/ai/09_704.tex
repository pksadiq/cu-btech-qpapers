%
% This file is part of Calicut University Question Paper Collection.
%
% Copyright (c) 2012-2015 Mohammed Sadik P. K. <sadiq (at) sadiqpk (d0t) org>.
% License: GNU GPLv3 or later
%
% Calicut University Question Paper Collection is free software: you can
% redistribute it and/or modify
% it under the terms of the GNU General Public License as published by
% the Free Software Foundation, either version 3 of the License, or
% (at your option) any later version.
% 
% Calicut University Question Paper Collection is distributed in the hope
% that it will be useful,
% but WITHOUT ANY WARRANTY; without even the implied warranty of
% MERCHANTABILITY or FITNESS FOR A PARTICULAR PURPOSE.  See the
% GNU General Public License for more details.
% 
% You should have received a copy of the GNU General Public License
% along with Calicut University Question Paper Collection.
% If not, see <http://www.gnu.org/licenses/>.
% 
%

\def \subj{AI 09 704---ANALOG AND DIGITAL CIRCUIT DESIGN}

\mainhead{D 50566}{2}

\semseven{NOVEMBER 2013}

\sub{\subj}

\maxtime

\partA

\iitem What does it mean by channel is pinched off?
\item How is it possible to eliminate feed forward path through Miller
  capacitor in frequency compensated circuits?
\item If the clock frequency of parallel switched capacitor
  equivalent resistor is 100kHz, find the value of the capacitor `C' that
  will emulate a 1M$\Omega$ resistor.
\item Name the different data objects in VHDL.
\item Check whether the following declarations are correct. If not, make
  necessary corrections:
  \iitem Signal CLOCK: BIT
  \item Variable COUNT: INTEGER
\ene

\markA
\partB

\item Draw the drain characteristic for a depletion mode transistor for
  different values of Vgs.
\item Explain switched capacitor integrator.
\item Draw the circuit diagram of CMOS sample and hold circuit.
\item Differentiate between EXIT and NEXT statements.
\item Implement a half adder using structural modelling.
\item What is the function of Assertion statement in VHDL?

\markB

\newpage \again
  
\partC

\item \iitem Draw and explain C-V characteristic of a MOS structure.
  \Or
\item Derive the gain and frequency response of a Differential Amplifier.
  \ene

\item \iitem Draw the circuit diagram and explain the operation of a
  folded cascode amplifier.
  \Or
\item Explain the MOS two stage amplifier in detail.
\ene
  
\item \iitem Explain the different types of iteration schemes of using loop
  statements in VHDL.
\Or
\item Write short notes on
\iitem Generics \item Configurations. \ene
\ene

\item \iitem Model a 4-bit adder using generate statement.
\Or
\item Model a 4-bit serial in serial out shift register.
  \ene
  \markC
  \ene

\mainhead{44469}{2}

\semseven{JUNE 2013 (Supplementary)}

\sub{\subj}

\maxtime


\partA


\iitem What is channel length modulation?

\item Give the advantages of folded cascode over simple cascode amplifier.

\item What is slew rate in Op-Amp?

\item Write the entity declaration for a 1-bit half adder in VHDL.

\item Give two examples for Extended identifiers.

\markA
\partB

\item Draw and explain the drain characteristic of MOSFET with varying V$_{\text{gs}}$.

\item Explain the capacitive model of MOS device.

\item Explain the operation of a two stage operational amplifier.

\item List the different categories of predefined operators in VHDL.

\item Differentiate between EXIT and NEXT statements.

\markB
\partC

\item \iitem \iitem Derive the equation for threshold voltage of a MOS device.

\item Draw and explain cascode current mirror circuit.

\ene

\Or

\item Derive the gain and frequency response of a Differential Amplifier.

\ene

\newpage \again

\item \iitem Explain the different frequency compensation in Operational Amplifiers.

\Or

\item Draw the circuit diagram and explain the operation of a folded cascode amplifier.

\ene

\item \iitem Explain the different types of modelling styles of architecture body in VHDL with examples.

\Or

\item Describe the different data types in VHDL.

\ene

\item \iitem Implement structural and behavioural models of a 2:4 decoder in VHDL.

\Or

\item Model an up/down counter using structural modelling in VHDL.

\ene

\markC

\ene


\newpage



\mainhead{29763}{2}

\semseven{OCTOBER 2012}

\sub{\subj}

\maxtime


\partA


\iitem What is body effect?

\item State the difference between current source and current mirror.

\item What is charge feedthrough?

\item What are the primary design units in VHDL?

\item Write the architecture body for a tristate buffer using behavioral modeling.


\markA
\partB

\item Briefly explain the Cascode current mirror and its advantages.

\item Explain the frequency compensation in Operational Amplifiers.

\item Draw the schematic of a switched capacitor integrator and explain it operation.

\item Briefly explain the various classes of Data types in VHDL.

\item Explain the generate and generic statements in VHDL with suitable examples.

\item Implement the following expressing using a suitable PLA
      
\hspace{2cm}  F (A, B, C) = $\sum m$($0, 2, 3, 5, 7$)

\markC
\partC

\item \iitem With neat sketch explain the MOS device structure.

\Or

\newpage \again

\item Derive the gain and frequency response of a Differential amplifier.

\ene

\item \iitem With circuit schematic explain the operation of folded cascode amplifier.

\Or

\item \iitem With schematic explain the operation of a summing amplifier and derive an
  expression for its output.

\item Explain the operation of a two stage operational amplifier.

\ene\ene

\item \iitem Write the VHDL code for Full Adder using (i) Data Flow and (ii) Structural Modeling.

\Or

\item With suitable examples, explain the Assertion and Report statements with their default severity
  levels.

\ene

\item \iitem Write the VHDL code for a $8 \times 1$ Multiplexer using select and case statements in VHDL.

\Or

\item Write the VHDL code for a $ 4 \times 4$ array multiplier using structural modeling in VHDL.

\ene
\markC
\ene


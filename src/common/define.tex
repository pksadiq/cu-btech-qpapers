%
% This file is part of Calicut University Question Paper Collection.
%
% Copyright (c) 2012-2015 Mohammed Sadik P. K. <sadiq (at) sadiqpk (d0t) org>.
% License: GNU GPLv3 or later
%
% Calicut University Question Paper Collection is free software: you can
% redistribute it and/or modify
% it under the terms of the GNU General Public License as published by
% the Free Software Foundation, either version 3 of the License, or
% (at your option) any later version.
% 
% Calicut University Question Paper Collection is distributed in the hope
% that it will be useful,
% but WITHOUT ANY WARRANTY; without even the implied warranty of
% MERCHANTABILITY or FITNESS FOR A PARTICULAR PURPOSE.  See the
% GNU General Public License for more details.
% 
% You should have received a copy of the GNU General Public License
% along with Calicut University Question Paper Collection.
% If not, see <http://www.gnu.org/licenses/>.
% 
%

% This file contain commands or function definitions used widely
% in this book (Calicut University B.Tech Question Paper)

\newcounter{qpage}
\newcounter{pager}
\newcounter{tmppage}
\newcommand{\mainhead}[2]{

\ifthenelse {\isodd{\thepage}}{}{\thispagestyle{empty}\char255\newpage

}


\def \pcode{#1}
\def \tpage{#2}
\setcounter{pager}{\tpage}
\setcounter{qpage}{0}                           % FIXME: Now its a simple hack
\setcounter{tmppage}{\thepage}
\fancyhead[L]{
\begin{tabular}{@{}l@{}}
\Large{\textbf{#1}}\\\ 
\end{tabular}}

\ifthenelse {\tpage>1}{
\fancyfoot[R]{\textbf{Turn over}}}{}

\fancyhead[C]{
\begin{tabular}{@{}c@{}}
{\textbf{{(\text{Pages\hspace{0.1cm}:\hspace{0.1cm}\href{http://www.sadiqpk.org}{#2}})}}}\\\ 
\end{tabular}}

\fancyhead[R]{
\begin{tabular}{@{}l@{}}
{\textbf{Name: ...............................}}\\
{\textbf{Reg. No: ...........................}}
\end{tabular}}

}


\newcommand{\again}
{

\fancyhf{}
\fancyhead[R]{\textbf{\pcode}}
\stepcounter{qpage}
\fancyhead[C]{\href{http://www.sadiqpk.org}{\theqpage}}

\ifthenelse{\isodd{\theqpage}}{
\ifthenelse{\value{pager}>\value{qpage}}{
\fancyfoot[R]{\textbf{Turn over}}}{}}{}
}


\newcommand{\ud}{\,\mathrm{d}}


\newcommand{\sectionall}{
\begin{center}
{\em Section I (Basics of Civil Engineering) and Section II (Basics of Mechanical Engineering)\\
are to be answered in} \textbf{separate} {\em answer books.\\
Assume suitable data wherever necessary.}
\end{center}
}


\newcommand{\separate}{
\begin{center}
{\em Answer Section A and B in }\textbf{separate }{\em answer books.}
\end{center}}


\newcommand{\sectionA}{

\begin{center}
\textbf{Section I (Basics of Civil Engineering)}
\end{center}}


\newcommand{\sectionB}{

\begin{center}
\textbf{Section II (Basics of Mechanical Engineering)}
\end{center}}


\newcommand{\sectionC}{

\begin{center}
\textbf{Section I (Basics of Electrical Engineering)}
\end{center}}


\newcommand{\sectionD}{

\begin{center}
\textbf{Section II (Basics of Electronics and Communication Engineering)}
\end{center}}


\newcommand{\sectionEE}{

\begin{center}
\textbf{Section A (Engineering Economics)}
\end{center}}

\newcommand{\sectionPM}{

\begin{center}
\textbf{Section B (Principles of Management)}
\end{center}}


\newcommand{\iitem}{
\begin{enumerate}
\item }

\newcommand{\ene}{

\end{enumerate}
\ifthenelse{\thetmppage < \thepage}{\again \stepcounter{tmppage}}{}
}


\newcommand{\comb}[1]{
\begin{center}
{ \textbf{COMBINED FIRST AND SECOND SEMESTER B.TECH. (ENGINEERING)}}\\
{ \textbf{DEGREE EXAMINATION, #1}}
\end{center}
}


\newcommand{\semthree}[1]{
\begin{center}
{ \textbf{THIRD SEMESTER B.TECH. (ENGINEERING) DEGREE EXAMINATION}}\\
{ \textbf{#1}}
\end{center}
}


\newcommand{\semfour}[1]{
\begin{center}
{ \textbf{FOURTH SEMESTER B.TECH. (ENGINEERING) DEGREE EXAMINATION}}\\
{ \textbf{#1}}
\end{center}
}


\newcommand{\semfive}[1]{
\begin{center}
{ \textbf{FIFTH SEMESTER B.TECH. (ENGINEERING) DEGREE EXAMINATION}}\\
{ \textbf{#1}}
\end{center}
}


\newcommand{\semsix}[1]{
\begin{center}
{ \textbf{SIXTH SEMESTER B.TECH. (ENGINEERING) DEGREE EXAMINATION}}\\
{ \textbf{#1}}
\end{center}
}


\newcommand{\semseven}[1]{
\begin{center}
{ \textbf{SEVENTH SEMESTER B.TECH. (ENGINEERING) DEGREE EXAMINATION}}\\
{ \textbf{#1}}
\end{center}
}


\newcommand{\semeight}[1]{
\begin{center}
{ \textbf{EIGHTH SEMESTER B.TECH. (ENGINEERING) DEGREE EXAMINATION}}\\
{ \textbf{#1}}
\end{center}
}


\newcommand{\sembt}[2][\sem]{  
\begin{center}
{\large\textbf{#1 SEMESTER B.TECH. (ENGINEERING) DEGREE}}\\
{\large\textbf{EXAMINATION, #2}}
\end{center}
}


\newcommand{\sub}[2][2009]{

\begin{center}
{ #2\\
(#1 Admissions)
}
\end{center}
}


\newcommandtwoopt{\maxtime}[2][Three][70]{
\noindent \begin{tabular}{@{}p{0.5\textwidth}>{\raggedleft\arraybackslash} p{0.5\textwidth}}
Time: #1 Hours & {      Maximum: #2 Marks}
\end{tabular}
}


\newcommand{\partA}{

\begin{center}
%\def \cval{est}
\textbf{Part A}\\
{\em Answer }\textbf{all} {\em questions}
\end{center}

}


\newcommand{\partB}{

\suspend{enumerate}

\begin{center}
\textbf{Part B}\\
{\em Answer any} \textbf{four} {\em questions}.
\end{center}
%\cval
\resume{enumerate}

%\setcounter{enumi}{\cval}

}

\newcommand{\partBt}{

\suspend{enumerate}

\begin{center}
\textbf{Part B}\\
{\em Answer any} \textbf{two} {\em questions}.
\end{center}
%\cval
\resume{enumerate}

%\setcounter{enumi}{\cval}

}


\newcommand{\partC}{

\suspend{enumerate}
\begin{center}
\textbf{Part C}\\
{\em Answer} \textbf{all} {\em questions}.
\end{center}
\resume{enumerate}

}


\newcommand{\partCo}{

\suspend{enumerate}
\begin{center}
\textbf{Part C}\\
{\em Answer section }($a$) \textbf{or} {\em section }($b$) {\em of each question}.
\end{center}
\resume{enumerate}

}


\newcommand{\marko}[1]{
%\suspend{enumerate}
%\suspend{enumerate}
%\begin{adjustwidth}{-1cm}{-1cm}
\hfill  \char255\hspace{-20cm}\begin{tabular}{@{}>{\raggedleft\arraybackslash} p{\textwidth}@{\hspace{-0.6cm}}}
\raggedleft{(#1 marks)}
\end{tabular}
%\end{adjustwidth}
%\resume{enumerate}\vspace{1cm}
%\resume{enumerate}
}


\newcommand{\marka}{
%\suspend{enumerate}
%\suspend{enumerate}
%\begin{adjustwidth}{-1cm}{-1cm}
\hfill  \char255\hspace{-20cm}\begin{tabular}{@{}>{\raggedleft\arraybackslash} p{\textwidth}@{\hspace{-0.5cm}}}
\raggedleft{(1 mark)}
\end{tabular}
%\end{adjustwidth}
%\resume{enumerate}\vspace{1cm}
%\resume{enumerate}
}


\newcommand{\markA}{
\ifthenelse{\lengthtest{\parindent>0pt}}{
\def \more{\leftmargin}}{\def \more{17.62482pt}}
\begin{tabular}{@{\hspace{1pt}}>{\raggedleft\arraybackslash} p{\textwidth - \more}@{}}
\raggedleft{(5 $\times$ 2 $=$ 10 marks)}
\end{tabular}
}


\newcommand{\markB}{

\ifthenelse{\lengthtest{\parindent>0pt}}{
\def \more{\leftmargin}}{\def \more{17.62482pt}}
\begin{tabular}{@{\hspace{1pt}}>{\raggedleft\arraybackslash} p{\textwidth -\more}@{}}
\raggedleft{(4 $\times$ 5 $=$ 20 marks)}
\end{tabular}

}


\newcommand{\markC}{
\ifthenelse{\lengthtest{\parindent>0pt}}{
\def \more{\leftmargin}}{\def \more{17.62482pt}}
\begin{tabular}{@{\hspace{1pt}}>{\raggedleft\arraybackslash} p{\textwidth -\more}@{}}
\raggedleft{(4 $\times$ 10 $=$ 40 marks)}
\end{tabular}
}


\newcommand{\markany}[3]{
\begin{tabular}{@{}>{\raggedleft\arraybackslash} p{\textwidth}@{}}
\raggedleft{(#1 $\times$ #2 $=$ #3 marks)}
\end{tabular}
}



\newcommand{\Or}{

\suspend{enumerate}
\begin{center}
{\em Or}
\end{center}
\resume{enumerate}
}

\newcommand{\enu}[1]{
\setcounter{enumi}{#1-1}}


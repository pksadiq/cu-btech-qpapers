%
% This file is part of Calicut University Question Paper Collection.
%
% Copyright (c) 2012-2015 Mohammed Sadik P. K. <sadiq (at) sadiqpk (d0t) org>.
% License: GNU GPLv3 or later
%
% Calicut University Question Paper Collection is free software: you can
% redistribute it and/or modify
% it under the terms of the GNU General Public License as published by
% the Free Software Foundation, either version 3 of the License, or
% (at your option) any later version.
% 
% Calicut University Question Paper Collection is distributed in the hope
% that it will be useful,
% but WITHOUT ANY WARRANTY; without even the implied warranty of
% MERCHANTABILITY or FITNESS FOR A PARTICULAR PURPOSE.  See the
% GNU General Public License for more details.
% 
% You should have received a copy of the GNU General Public License
% along with Calicut University Question Paper Collection.
% If not, see <http://www.gnu.org/licenses/>.
% 
%

\def \subj{ AI 09 605---INDUSTRIAL INSTRUMENTATION}

\mainhead{C 41263}{2}
\semsix{MAY 2013}
\sub{\subj}
\maxtime

\partA

\iitem State Peltier effect.
\item Where are bimetallic elements used as temperature sensors?
\item Why elastic element type gauges are preferred over liquid
  column manometers in industry?
\item Define Reynolds number.
\item Differentiate between float type and displacer type liquid level gauges.

\markA
\partB

\item How is liquid level in a boiler drum measured?
\item A resistance thermometer is made of nickel wire. Thermometer resistance at 20$^\circ$C is 100$\Omega$.
  If the resistivity is 0.8m$\Omega$m, what would be length of the wire if 2 mm diameter wire is used.
  f the resistance varies linearly with temperature, what would be the resistance at temperatures
  $ t = -50^\circ$C and $ t = 100^\circ$C? Assume sensitivity as 0.2$\Omega$/$^\circ$C.
\item Why is cold junction compensation necessary in temperature measuring schemes using 
  thermocouples? Discuss a recent trend in making such compensation.
\item Explain working principle of a thermal conductivity gauge for low pressure measurement.
\item Discuss how a dead weight tester can be employed for pressure calibration.
\item Explain the working of a positive displacement type flow meter.

\markB
\partC

\item \iitem Explain the working principle of an optical pyrometer with a neat block diagram.

\newpage \again

\Or
\item Discuss the working of RTD. How temperature can be measured using a 4-lead RTD
  arrangement.
\ene

\item \iitem Describe in detail about the principle of an elastic type Bellows element with suitable sketches.
\Or
\item Explain the working of a capacitive type differential pressure transmitter.
\ene

\item \iitem Explain the principle of operation of a rotameter.
\Or
\item Describe a mass flow meter that uses the principle of conservation of 
  angular momentum.
\ene

\item \iitem Discuss the working principle of a ultrasonic flow meter.
\Or
\item Describe a technique for the measurement of liquid level or solid level using radioactive
  sources and detectors.
\ene

\markC
\ene

\newpage

\mainhead{C 26776}{2}
\semsix{MAY 2012}
\sub{\subj}
\maxtime

\partA

\iitem Define calibration.
\item What are force summing devices?
\item State Hooks Law.
\item State any two flow characteristics.
\item What is an anemometer?

\markA
\partB

\item Explain the principle of operation of a Thermocouple.
\item Write a note on Quartz Crystal Thermometer.
\item The output voltage of an LVDT is 1.5V at maximum displacement. At a load of
  0.5M$\Omega$, the deviation from linearity is maximum and it is $\pm$0.003V from a straight line
  through origin. Find the linearity at the given load.
\item With neat sketch explain the Bellows and Diaphragms.
\item Explain the tapping and piping arrangements for the Flow Meter installation.
\item Write a note on Cross Correlation and its applications in Flow Measurement.

\markB

\newpage \again

\partC

\item \iitem Briefly explain the reference junction consideration for a Thermocouple. \marko{4}
\item Explain the Operation of a Pressure Thermometer.  \marko{6}
\ene
\Or
\item Discuss in detail about the construction, working and applications of RTDs.

\item Explain the operations of Strain Gauge. Derive an expression for the Gauge Factor of a Strain Gauge.
\Or
\item Explain in detail about the High Pressure and Low Pressure Measurements.

\item Explain the construction and working of a Head Flow meter.
\Or
\item \iitem Explain the working of a Electromagnetic Flow meter and Vortex Flow meter. \marko{6}
\item Briefly explain the Turbine type Flow Meter.  \marko{4}
\ene

\item Discuss in detail about the Ultrasonic Flow Meter.
\Or
\item \iitem Explain the various types of Floats. \marko{4}
\item Discuss in detail about the various types of Level Measurements. \marko{6}
\ene

\markC
\ene

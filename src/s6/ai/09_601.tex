%
% This file is part of Calicut University Question Paper Collection.
%
% Copyright (c) 2012-2015 Mohammed Sadik P. K. <sadiq (at) sadiqpk (d0t) org>.
% License: GNU GPLv3 or later
%
% Calicut University Question Paper Collection is free software: you can
% redistribute it and/or modify
% it under the terms of the GNU General Public License as published by
% the Free Software Foundation, either version 3 of the License, or
% (at your option) any later version.
% 
% Calicut University Question Paper Collection is distributed in the hope
% that it will be useful,
% but WITHOUT ANY WARRANTY; without even the implied warranty of
% MERCHANTABILITY or FITNESS FOR A PARTICULAR PURPOSE.  See the
% GNU General Public License for more details.
% 
% You should have received a copy of the GNU General Public License
% along with Calicut University Question Paper Collection.
% If not, see <http://www.gnu.org/licenses/>.
% 
%

\def \subj{ AI 09 601---DIGITAL SIGNAL PROCESSING}


\mainhead{C 61464}{2}
\semsix{APRIL 2014}
\sub{\subj}
\maxtime

\partA

\iitem State sampling theorem.
\item Define Discrete Fourier Transform of a signal.
\item What is an IIR filter? Give an example.
\item What is a causal system?
\item State the advantage of parallelism.

\markA
\partB

\item Discuss the relation of DFT with two other transforms.
\item Explain any three applications of FFT.
\item Write notes on signal flow graphs.
\item Assume an IIR filter. Draw the Direct Form I and Direct Form II
  structures of the filter.
\item Discuss the characteristics of Butterworth filter.
\item Explain pipelining.

\markB
\partCo

\item
  \iitem State and Prove the properties of DFT.

\newpage \again

  \Or
  \item Find the 4-point DFT of the sequence $x(n) = \{2, 1, 4, 3\}$ by:
    \iitem DIT FFT algorithm.
    \item DFT FFT algorithm.
    \ene
  Also plot the magnitude and phase plots.
  \ene

\item
  \iitem Explain the effects of quantization in digital structure realization.
  \Or
  \item Discus the following:
    \iitem Round off Error.
    \item Truncation Error.
    \item Limit Cycle oscillations.
    \ene
  \ene  

\item
  \iitem
    \iitem Explain the characteristics of FIR filters with linear phase.
    \item Convert H$(s) = \dfrac{2}{\{s(s + 2)\}}$ into H($z$) by impulse invariant
      method for a sampling frequency of 4 samples per second.
    \ene
\Or
\item \iitem Using Bilinear transformation method, obtain H$(z)$ from H($s$) when
  T = 0.5s.

  H$(s) = \dfrac{4}{\{ (s+3) (s+4)\}}$
\item Explain the design of FIR filter using window method.
\ene
\ene

\item \iitem \iitem Explain the special instructions used in DSP.
\item Explain the architecture of a General purpose DSP processor.\ene
\Or
\item \iitem Write notes on TMS320 series processors.
\item Explain the implementation of a multiplier. \ene
\ene

\markC
\ene

\newpage

\mainhead{C 41260}{2}
\semsix{MAY 2013}
\sub{\subj}
\maxtime

\partA

\iitem Show that, for symmetric $x$($n$), $n = 0, 1\ldots, $N - 1, the DFT X($k$) = 0, for $k$ = N/2.
\item Obtain the circular convolution of $x$[$n$] = \{1, 2, 1\} with $y$[$n$] = \{1, -1\}.
\item Draw the lattice structure realization of the FIR filter H($z$) = 1 + $\frac{1}{2}z^{-1}$.
\item Write the transformation equation to convert a digital low-pass filter into a digital high-pass
  filter.
\item What are the different buses in TMS 320 C 54 processor?

\markA
\partB

\item Show that 8-point DFT can be expressed in terms of two 4-point DFTs.
\item Let N-point DFT of $x$($n$) is X($k$). Express DFT of $x^*$($n$) and $e^{-j4\pi mn/\text{N}}x$($n$) in
  terms of X($k$).
\item What is overflow error? How is it prevented?
\item Prove that a stable analog filter will be mapped to a stable digital
  filter through impulse invariant transform.
\item Convert the analog filter having transfer function H($s$) = $\dfrac{1}{s^2 + 3s + 2}$ into digital
  IIR filter using impulse invariant method.
\item With an example explain how a specific DSP hardware will increase the processing speed of a 
  DSP algorithm implementation.

\markB
\partCo

\item \iitem \iitem State and prove convolution property of DFT. \marko{5}
\item Show that DFT of even part of a signal $x$($n$) is equal to the real part of
  the DFT of \\ $x$($n$). \marko{5}
\ene 

\newpage \again

\Or
\item \iitem Show that DFT of two real sequences of length N can be computed using one
  N-point DFT. \mark{6}
\item State and prove time shifting property of DFT. \mark{4}
\ene \ene

\item \iitem Draw the direct form and lattice-ladder from realizations of the IIR filter:
\[ \text{H(}z\text{)} = \dfrac{1 + 2z^{-1} + 3z^{-2} + 2^{-3}}{1 + 0.9z^{-1} - 0.8z^{-2} + 0.5z^{-3}}\].
\Or
\item Explain the limit cycle oscillations of a digital filter with respect the system described by the
  difference equation $y$[$n$]$ = 0.95y$[$n - 1$]$ + x$[$n$]. Also determine the dead band of the filter.
\ene
\item \iitem Design an FIR linear phase filter using Hamming window approximating the ideal frequency
  response:

\hspace{0.5cm} H($w$) = $\begin{cases}
  1, & \text{for }|w| \leq \frac{\pi}{4}\\
  0, & \text{for } \frac{\pi}{4} < |w| \leq \pi
\end{cases}$

Assume filter length L = 13. Draw the filter structure in Direct form.
\Or
\item Design a digital IIR filter with the following specifications:\\
  pass band = 0 - 12 kHz, stop band = 12.6 - 16 kHz, pass band ripple $<$ 0.1dB,
  stop band attenuation $>$ 30 dB, sampling frequency = 32 kHz. Draw the filter structure
  for the filter.
\ene

\item \iitem Describe the function of on chip peripherals of TMS 320 series processors.
\Or
\item What are the DSP specific processing units and instructions present in a typical digital
  signal processor? Explain with appropriate examples.
\ene

\markC
\ene

\newpage

\mainhead{C 26773}{2}
\semsix{MAY 2012}
\sub{\subj}
\maxtime

\partA

\iitem Show that DFT is a linear transform.
\item State the convolution property of the DFT.
\item Name any two errors occur due to finite word length effects in DSPs.
\item Define linear phase characteristics of a filter.
\item What is a MAC in a DSP chip?

\markA
\partB

\item Show that DFT of real signal is conjugate symmetric and DFT of a conjugate symmetric
  signal is real.
\item A data sequence having 10,000 samples is to be filtered using a 101 FIR filter. The filter
  is implemented in the DFT domain using 512 point DFT and overlap add method. Find out the percentage savings
  in computation (number of multiplications and additions required) compared to the direct implementation of
  convolution.
\item Draw the signal flow graph of direct form realizations (Type I and Type II) for the IIR filter
  represented by

\hspace{1cm}  $\dfrac{ 1 + 2 z^{-1} + 3 z^{-2} + 2z^{-3}}{
    1 + 0.9z^{-1} - 0.8z^{-2} + 0.5z^{-3}}$.
\item Using bilinear transform, determine the digital filter transfer function corresponding to the analog
  filter.
\item Discuss the architecture design enabling parallel processing in a Digital Signal Processor.

\hspace{2cm}$\text{H(s)} = \dfrac{8 + 0.1}{(8 + 0.1)^2 + 9 }.$ Take T = 1.

\markB

\newpage \again

\partCo
\item 
\iitem Show that linear convolution can be implemented through circular convolution and describe the 
  steps involved in implementing linear filtering in the DFT domain using overlap
  save method.
\Or
\item With appropriate signal flow graph, explain the implementation of decimation in time FFT 
  algorithm for computing an eight point DFT. Also determine the number of complex multiplications
  and additions 
  required for the DFT computation.
\ene

\item \iitem
  Draw the direct form realization and lattice structure realization of the FIR filter represented
  by the transfer function H($z$)$ = 1 + 2.88z^{-1} + 3.404 z^{-2} + 1.74 z^{-3} + 0.4 z^{-4}$.
\Or
\item Obtain the impulse response of the filter $y(n) = x(n) + 0.5y(n-1)$ assuming four bit
  representation for the coefficients and input $x(n).$
\ene

\item \iitem Determine the system function H(z) of the lowest-order Chebyshev digital filter that meets the following
  specifications:
\iitem 0.5 dB ripple in the pass-band $ 0 \leq |w| \leq 0.24\pi.$
\item At least 50 dB attenuation in the stop-band $0.35\pi \leq |w|\leq \pi.$ Draw the Direct form II
  implementation of the filter.
\ene
\Or
\item Design an FIR linear phase filter using frequency sampling method approximating the ideal frequency response.

\hspace{2cm}H($w$) $\begin{cases}
 1, & \text{for }  |w| \leq \dfrac{\pi}{6}\\
 0, &\text{for } \dfrac{\pi}{6} <  |w| \leq \pi
\end{cases}$

Assume filter length L = 11. Draw the filter structure in Direct form.
\ene


\item \iitem What are the distinct features of DSP processor architecture? Explain with an
  example processor architecture.
\Or
\item What is meant by instruction pipelining? Explain with an example how pipelining
  increases through put efficiency.
\ene

\markC
\ene

%
% This file is part of Calicut University Question Paper Collection.
%
% Copyright (c) 2012-2015 Mohammed Sadik P. K. <sadiq (at) sadiqpk (d0t) org>.
% License: GNU GPLv3 or later
%
% Calicut University Question Paper Collection is free software: you can
% redistribute it and/or modify
% it under the terms of the GNU General Public License as published by
% the Free Software Foundation, either version 3 of the License, or
% (at your option) any later version.
% 
% Calicut University Question Paper Collection is distributed in the hope
% that it will be useful,
% but WITHOUT ANY WARRANTY; without even the implied warranty of
% MERCHANTABILITY or FITNESS FOR A PARTICULAR PURPOSE.  See the
% GNU General Public License for more details.
% 
% You should have received a copy of the GNU General Public License
% along with Calicut University Question Paper Collection.
% If not, see <http://www.gnu.org/licenses/>.
% 
%

\def \subj{ AI 09 L01---WIRELESS COMMUNICATION SYSTEMS}

\mainhead{C 41264}{2}
\semsix{MAY 2013}
\sub{\subj}
\maxtime

\partA

\iitem Differentiate LEOs, MEOs and GEOs.
\item What is the `run property' of PRBs?
\item What is EDGE?
\item Determine the number of cells in clusters for the following values:
  $j=4, i = 2$ and $ j = 3, i =3$.
\item What is the PUSH process in WAP?

\markB
\partB

\item An earth is located at latitude 35$^\circ$N and longitude 100$^\circ$W. Calculate the antennalook
  angles for a satellite at 67$^\circ$W.
\item Briefly explain about WLL.
\item Explain Frequency hopped spread spectrum systems.
\item If a total of 33 MHz of bandwidth is allocated to particular FDD cellular telephone system which
  uses two 25 kHz simplex channels to provide full duplex voice and control channels, compute
  the number of channels available per cell if the system uses seven cell reuse. If 1 Mhz of the allocated
  spectrum is used for control channels, determine an equitable distribution of control channels
  and voice channels in each cell.
\item Explain the terms coherence time and coherence bandwidth.
\item Explain the Bluetooth protocol stack.

\markB

\newpage \again

\partCo

\item \iitem A satellite link operating at 14 GHz has receiver feeder losses of 1.5 dB and a free-space
  loss of 207 dB. The atmospheric absorption loss is 0.5 dB, and the antenna pointing loss is 0.5 dB.
  Depolarization losses may be neglected. Calculate the total  link loss of clear sky conditions.
\Or
\item With block schematic, explain various components of microwave radio station.
\ene

\item \iitem Draw the circuit of a 4 bit maximal length sequence generator using shift registers.
  Also find the sequence.
\Or
\item Briefly explain the 3G wireless networks.
\ene

\item \iitem A base station produces 40 Watts of power applied to a unity gain antenna with 900 MHz carrier
  frequency. The receiving antenna with unity gain is located at a distance of 5 km.
  What is the received power for free space path loss model in decibels?
\Or
\item Discuss various measures to improve coverage and capacity in cellular systems.
\ene

\item \iitem Explain briefly the type of multiple accessing technique used in GSM.
\Or
\item Give an overview of different 802.11x WLAN standards.
\ene

\markC
\ene

\newpage

\mainhead{C 26777}{2}
\semsix{MAY 2012}
\sub{\subj}
\maxtime

\partA

\iitem What is the microwave frequency range? Give any two of its applications.
\item What is a Satellite Look angle?
\item Define Jamming Margin.
\item Define Path Loss and Fading.
\item Compare TDMA and FDMA.

\markA
\partB

\item State Kepler's Laws.
\item Briefly explain the applications and advantages of Satellite Communication.
\item Explain Channel Capacity.
\item Write a note on Third Generation Wireless Networks.
\item Write a note on Co-Channel Interference and Adjacent Channel Interference.
\item Write a note on Cordless Systems.

\markB
\partC

\item Explain the various diversity reception techniques.
\Or
\item Discuss in detail about the Satellite Link Budget.

\newpage \again

\item Explain in detail about CDMA.
\Or
\item \iitem Explain the features of Second Generation Cellular Networks. \marko{5}
\item What is a Wireless Local Loop? Explain any two of its technologies.  \marko{5}
\ene

\item Derive the two Ray Ground Propagation Model to estimate the received Signal Strength.
\Or
\item \iitem How to improve the coverage and capacity of a Cellular System? Explain.  \marko{6}
\item Briefly explain the following terms: Dwell Time, MAHO.  \marko{4}
\ene

\item What are the design guidelines for WAP? Explain.
\Or
\item Design an architecture and location update procedure that integrates IS-41 and GSM MAP.

\markC
\ene

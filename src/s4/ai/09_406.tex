%
% This file is part of Calicut University Question Paper Collection.
%
% Copyright (c) 2012-2015 Mohammed Sadik P. K. <sadiq (at) sadiqpk (d0t) org>.
% License: GNU GPLv3 or later
%
% Calicut University Question Paper Collection is free software: you can
% redistribute it and/or modify
% it under the terms of the GNU General Public License as published by
% the Free Software Foundation, either version 3 of the License, or
% (at your option) any later version.
% 
% Calicut University Question Paper Collection is distributed in the hope
% that it will be useful,
% but WITHOUT ANY WARRANTY; without even the implied warranty of
% MERCHANTABILITY or FITNESS FOR A PARTICULAR PURPOSE.  See the
% GNU General Public License for more details.
% 
% You should have received a copy of the GNU General Public License
% along with Calicut University Question Paper Collection.
% If not, see <http://www.gnu.org/licenses/>.
% 
%

\def \subj{ AI 09 406---ELECTRONIC INSTRUMENTATION AND MEASUREMENT}

\mainhead{C 26899}{2}
\semfour{MAY 2012}
\sub{\subj}
\maxtime

\partA

\iitem What is meant by systematic error?
\item Name the basic blocks of a digital instrumentation system.
\item What is the necessity of curve fitting?
\item What are the advantages of flash ADC? 
\item What do you mean by baby sitting mode in a DSO?

\markA
\partB

\item Define the terms:
\iitem Threshold
\item Hysteresis
\ene
\item A 0--100V voltmeter has 200 scale divisions which can be read to $\dfrac{1}{2}$ division. Determine the 
  resolution of the meter in volt.
\item Explain the impulse response of a first order system.
\item With schematic explain R--2R ladder DAC.
\item Write a note on LCD.
\item What is the basic principle of sampling oscilloscope.

\markB

\newpage \again

\partC

\item
\iitem With block schematic explain the functional elements of a measuring system.
\Or
\item Explain:
\iitem Line fitting.
\item Curve fitting.
\item Goodness of fit test.
\ene
\ene

\item 
\iitem With block diagram explain a pulse generator.
\Or
\item With block schematic describe the working of sweep frequency generator.
\ene

\item
\iitem With schematic explain Successive Approximation ADC.
\Or
\item Explain the principle of weighted capacitor DAC. Compare its performance with weighted resistor DAC.
\ene

\item
\iitem With block schematics explain true RMS meter.
\Or
\item Explain the principle of operation of X--Y recorder with the help of block diagram.
\ene

\markC
\ene

\newpage

\mainhead{15659}{1}
\semfour{JUNE 2011}
\sub{\subj}
\maxtime

\partA

\iitem Define calibration.
\item State the difference between accuracy and precision.
\item What is a unit? Name the various types of units.
\item What is an Integral Non linearity error?
\item What is a Q meter? Give any one of its application.

\markA
\partB

\item Write a note on Analog and digital modes of an instrument.
\item What is curve fitting? Explain.
\item Briefly explain the primary and working standards.
\item Explain the operation of a R--2R Ladder type DACs.
\item What is the principle of operation of an LCD? Explain.
\item What is RMS value? How is it measured using True RMS meter? Explain.

\markB

\newpage \again

\partC

\item  With block diagram, explain the generalized configuration of an Instrument.
\Or
\item With neat sketch explain the following type of Instruments:
\iitem Null type
\item Deflection type
\ene

\item  Discuss the first and second order instrument performance and their response
  to standard test signals.
\Or
\item \iitem How the system parameters are measured? Explain.
\item With block diagram, explain the function of a Digital Instrument.
\ene 

\item Explain the operation of a
\iitem Flash type ADC
\item Integrating type ADC
\ene
\Or
\item Discuss the operation of a
\iitem Bipolar DAC
\item Master-slave DAC.
\ene

\item  With block diagram, explain the operation of a CRO and derive an
  expression for its deflection sensitivity.
\Or
\item With block diagram, explain the operation of a spectrum analyzer and its applications.

\markC
\ene

%
% This file is part of Calicut University Question Paper Collection.
%
% Copyright (c) 2012-2015 Mohammed Sadik P. K. <sadiq (at) sadiqpk (d0t) org>.
% License: GNU GPLv3 or later
%
% Calicut University Question Paper Collection is free software: you can
% redistribute it and/or modify
% it under the terms of the GNU General Public License as published by
% the Free Software Foundation, either version 3 of the License, or
% (at your option) any later version.
% 
% Calicut University Question Paper Collection is distributed in the hope
% that it will be useful,
% but WITHOUT ANY WARRANTY; without even the implied warranty of
% MERCHANTABILITY or FITNESS FOR A PARTICULAR PURPOSE.  See the
% GNU General Public License for more details.
% 
% You should have received a copy of the GNU General Public License
% along with Calicut University Question Paper Collection.
% If not, see <http://www.gnu.org/licenses/>.
% 
%

\def \subj{AI 09 404---INTRODUCTION TO MICROPROCESSORS}

\mainhead{C 26897}{2}
\semfour{MAY 2012}
\sub{\subj}
\maxtime

\partA

\iitem What are low level language and high level language?
\item Write a program to turn on the air conditioner if switch S$_6$ of the input port O/H is on. Ignore all
  other switches of the input port even if someone attempts to turn on other applications.
\item If CS = 123A and IP = 341B, determine the physical address of the next instruction.
\item What is Interrupt on terminal count?
\item Write the control word to program 8255 in model with port B as input port.

\markA
\partB

\item Explain the function of the signals RESETOUT, HOLD, READY and TRAP.
\item Write a program to count the numbers of 1's in the given data.
\item Differentiate between the minimum and maximum mode of 8086.
\item With an example, how I/O devices and memory devices are interfaced with 8085.
\item Write a program to convert a Binary number to ASCII.
\item Explain about ICW and OCW of 8259.

\markB
\partC

\item \iitem Discuss the characteristics of Large, Medium and Microcomputers. \marko{7}
\item Explain about Instruction format and flag register of 8085. \marko{3}
\ene
\Or
\item Describe the block diagram of 8085.

\item Write a program to arrange 10 numbers in ascending order.
\Or
\item Write a program to convert a BCD number to seven-segment display.

\newpage \again

\item 
\iitem Write an ALP to find the average of N numbers. \marko{5}
\item Draw and explain the timing diagram of a memory read cycle. \marko{5}
\ene
\Or
\item Explain the architecture of 8086.

\item Discuss in detail about 8257 DMA controller.
\Or
\item Describe the features and operation of 8279.

\markC
\ene

\newpage

\mainhead{C 15657}{1}
\semfour{JUNE 2011}
\sub{\subj}
\maxtime

\partA

\iitem What are the addressing modes of 8085?
\item Write a program to generate a time delay of 1 m. sec.
\item What is the function of BIV and EV.
\item What is meant by programmable one shot in 8283 timer?
\item What is pipelining?

\markA
\partB

\item Explain how a DMA transfer is carried out in 8085.
\item Write a program to multiply two 8 bit numbers.
\item Explain about memory segmentation in 8086.
\item Explain the mode 0 operation of 8255 with an example.
\item Explain the principle behind A/D converter.
\item Explain about Assemblers.

\markB
\partC

\item Discribe the architecture of 8085 microprecessor.
\Or
\item Discuss the memory organisation of a microprocessor and also explain how I/O devies are interfaced with
  the microprocessor.

\newpage\again

\item Write a program to convert a binary number to a BCD number.
\Or
\item Write a program to find the average of N numbers.

\item Explain the addressing modes of 8086.
\Or
\item Discuss about the signals present in 8086 processor.

\item Describe the operation of 8259.
\Or
\item Explain how keyboard/display controller works.

\markC
\ene

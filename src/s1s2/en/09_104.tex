%
% This file is part of Calicut University Question Paper Collection.
%
% Copyright (c) 2012-2015 Mohammed Sadik P. K. <sadiq (at) sadiqpk (d0t) org>.
% License: GNU GPLv3 or later
%
% Calicut University Question Paper Collection is free software: you can
% redistribute it and/or modify
% it under the terms of the GNU General Public License as published by
% the Free Software Foundation, either version 3 of the License, or
% (at your option) any later version.
% 
% Calicut University Question Paper Collection is distributed in the hope
% that it will be useful,
% but WITHOUT ANY WARRANTY; without even the implied warranty of
% MERCHANTABILITY or FITNESS FOR A PARTICULAR PURPOSE.  See the
% GNU General Public License for more details.
% 
% You should have received a copy of the GNU General Public License
% along with Calicut University Question Paper Collection.
% If not, see <http://www.gnu.org/licenses/>.
% 
%

\def \subj{EN 09 104/PTEN 09 104---ENGINEERING CHEMISTRY}

\mainhead{C 40922}{2}
\comb{APRIL 2013}
\sub{\subj}
\maxtime

\partA

\iitem Comment on the effect of temperature on conductivity.
\item Write the chemical structures of EDTA and EBT.
\item Write the monomers of nylon-6, 6.
\item What is an electrochemical cell?
\item Define BOD and COD.

\markA
\partB

\item Write a short note on carbon nanotubes and nanowires.
\item How is hardness of water sample experimentally determined by EDTA
  method?
\item Write a note on any \emph{two} polymerization techniques.
\item Briefly discuss the determination and importance of aniline point
  and corrosion stability of lubricants.
\item The copper rods are placed in copper sulphate solution of concentration
  0.1 M and 0.01 M respectively to form a cell. Give the cell representation
  and calculate its EMF at 298 K.
\item Explain how metallic coatings are obtained by galvanizing.

\markB
\partCo

\item \iitem What are intrinsic and extrinsic semiconductors? Explain the
  semiconductivity in stoichiometric and non-stoishiometric compounds.

\newpage \again

\Or
\item Discuss in detail the various steps involved in the purification of
  water for domestic use.
\ene

\item \iitem \iitem What are the classifications of polymers? Give one
  example for addition and condensation polymers. \marko{5}
\item Bring out the importance of viscosity and corrosion stability
  of lubricants. \marko{5}
\ene
\Or
\item Explain the preparation, properties and structure PE, PS bakelite and
  silicons.
\ene

\item \iitem Discuss the construction, functioning and applications of fuel
  cells and solar cells.
\Or
\item \iitem How is pH of an acid solution determined by glass electrodes?
 \marko {5}
\item Derive Nernst equation. \marko{5}
\ene
\ene

\item \iitem Explain the resting of iron with the help of electrochemical
  theory of corrosion.
\Or
\item What are the sources of air pollution? Explain the harmful effects of
  ozone depletion and acid rain.
\ene

\markC
\ene

\newpage

\mainhead{C 26473}{2}
\comb{APRIL 2012}
\sub{\subj}
\maxtime

\partA

\iitem Differentiate between Intrinsic and Extrinsic semiconductors.
\item Why buffer is added during titration of hard water against EDTA?
\item Graphite functions as a lubricant. Why?
\item Write the redox reactions taking place in the lead acid accumulators.
\item Mention any {\em two} alternate refrigerants.

\markA
\partB

\item What are liquid crystals? Mention their applications in displays and thermography.
\item 0.30 g of $\text{CaCO}_3$ was dissolved in HCl and the solution made on to one litre
  with distilled water. 100 ml
  of this solution required 30 ml of EDTA solution on titration. 100 ml of hard water sample
  required 55 ml of same EDTA solution on 
  titration. After boiling 100 ml of this water, cooling, filtering and then titration 10 ml
  of EDTA solution. Calculate the temporary
  and permanent hardness of water.
\item Explain the structure relation to properties of polymers.
\item Enumerate the applications of polymers in electrical and electronic industry.
\item Write a short note on buffer solution. Express Henderson equation for calculation of pH.
\item What are the causes and consequences of thermal pollution?

\markB
\partCo

\item \iitem \iitem Explain the process of ultrapure silicon production. \marko{5}

\newpage \again

\item With relevant chemical equations, outline the estimation of dissolved oxygen
  volumetrically. \marko{5}
\ene
\Or
\item  Enumerate the various stages involved in the purification of water for domestic use.
\ene

\item \iitem Explain cationic, anionic and free radical mechanism of polymerization.
\Or
\item Discuss in detail the properties of lubricants highlighting their importance.
\ene

\item \iitem \iitem How is EMF of an electrochemical cell determined through
  Poggendorf's compensation method? \marko{5}
\item What is electrochemical series? What are its applications? \marko{5}
\ene
\Or
\item Write a descriptive account on Fuel cells and solar cells.
\ene

\item \iitem \iitem What is the mechanism of drying of oil plants? \marko{5}
\item Write a note on experimental determination of BOD of a polluted water sample. 

\marko{5}
\ene
\Or
\item Describe the effects of air pollution on environment. What are the methods employed
  to control air pollution?
\ene

\markC
\ene

\newpage

\mainhead{C 15006}{2}
\comb{MAY 2011}
\sub{\subj}
\maxtime

\partA

\iitem Differentiate between $n-$type and $p-$type semiconductors.
\item A sample of water contains the following impurities :
  \[ \text{Mg }\left(\text{HCO}_3\right)_2 = 75 \text{ mg/L, CaCl}_2 =
  278\text{ mg/L and MgSO}_4 = \text{142 mg/L.}\]

  Calculate the temporary and permanent hardness.
\item Name the monomers of Bakelite.
\item What is an hydrogen electrode?
\item What is direct corrosion?

\markA
\partB

\item Explain the classification and applications of liquid crystals.
\item How is water softened by lime soda process?
\item Briefly discuss the applications of polymers in electrical and electronic industry.
\item Derive the expression for EMF in concentration cells.
\item How is pH measured using glass electrode?
\item Discuss differential aeration corrosion with an example.

\markB

\newpage \again

\partCo

\item
  \iitem 
    \iitem Write a short note on ultrapure silicon production.
    \item Give the BIS specification of drinking water.
\ene
\Or
\item How is hardness of water determined experimentally by EDTA method?
\ene


\item
\iitem With a neat diagram, discuss the demineralization of water using ion exchange method.
\Or
\item Explain the cationic, anionic and free radical mechanisms of polymerization reactions.
\ene

\item     
  \iitem 
    \iitem Explain the theory of extreme pressure lubrication. \marko{5}
    \item Derive the expression for single electrode potential. \marko{5}
    \ene
\Or
\item Describe the construction and functioning of lead acid accumulators and Ni-Cd cells.
\ene

\item
  \iitem Briefly discuss on galvanic series and galvanic corrosion.
\Or
\item
\iitem Explain the cause and consequence of photochemical smog. \marko{5}
\item What is thermal pollution? What are its effects? \marko{5}
\ene
\ene

\markC
\ene

\newpage

\mainhead{C 6286}{2}
\comb{MAY 2010}
\sub{\subj}
\maxtime

\partA

\iitem What are semiconductors? Give an example for $n-$type semiconductors.
\item What is the hardness of water? How is it expressed?
\item Given {\em one} example each for thermoplastic and thermosetting polymers.
\item Define reduction potential and oxidation potential.
\item Define Pilling-Bed Worth rule.

\markA
\partB

\item Write about the applications of carbon nanotubes and nanowires.
\item How is hardness of water sample estimated through EDTA titration?
\item Discuss the mechanism of cationic polymerization.
\item What are synthetic rubbers? Give the preparation and structure of any {\em two} synthetic rubbers.
\item The potential of a hydrogen gas electrode in a solution of an acid of unknown strength is 0.29 V at 
  298 K as measured against normal hydrogen electrode. Calculate the pH of acid solution.
\item Give an account on photochemical smog and ozone depletion.

\markB
\partCo

\item \iitem Discuss all aspects of electrical conductivity in solids based on band theory.
\Or
\item Bring out the various steps involved in the purification of water for domestic use.
\ene

\newpage \again

\item \iitem Explain the structure relation to properties of polymers. Discuss the process and applications
  of vulcanization.
\Or
\item Explain thin film mechanism of lubrication. Discuss any {\em four} properties of lubricants.
\ene

\item \iitem What are fuel cells? Explain the construction and applications of
  $\text{H}_2/\text{O}_2$ fuel cells.
\Or
\item \iitem Derive Nernst equation. \marko{5}
\item Write a short note on solar cells. \marko{5}
\ene
\ene

\item \iitem Explain the mechanism of wet corrosion. Give details of corrosion protection
  through sacrificial anodic method and impressed current method.
\Or
\item \iitem How are metals protected from corrosion by electroplating?  \marko{5}
\item Write a short account on thermal pollution?  \marko{5}
\ene\ene

\markC
\ene

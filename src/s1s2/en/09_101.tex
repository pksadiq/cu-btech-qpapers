%
% This file is part of Calicut University Question Paper Collection.
%
% Copyright (c) 2012-2015 Mohammed Sadik P. K. <sadiq (at) sadiqpk (d0t) org>.
% License: GNU GPLv3 or later
%
% Calicut University Question Paper Collection is free software: you can
% redistribute it and/or modify
% it under the terms of the GNU General Public License as published by
% the Free Software Foundation, either version 3 of the License, or
% (at your option) any later version.
% 
% Calicut University Question Paper Collection is distributed in the hope
% that it will be useful,
% but WITHOUT ANY WARRANTY; without even the implied warranty of
% MERCHANTABILITY or FITNESS FOR A PARTICULAR PURPOSE.  See the
% GNU General Public License for more details.
% 
% You should have received a copy of the GNU General Public License
% along with Calicut University Question Paper Collection.
% If not, see <http://www.gnu.org/licenses/>.
% 
%

\def \subject {PTEN/EN09 101---ENGINEERING MATHEMATICS---I}

\mainhead{C 40919}{2}
\comb{APRIL 2013}
\sub{\subject}
\maxtime

\partA

\iitem Give the centre of curvature formula in cartesian form.
\item What is meant by Absolute convergence? Define.
\item Test for convergence the series $\sum\!\! \left\lgroup 1 +
  \dfrac{1}{\sqrt{n}}\right\rgroup^{\!-n^{^3\!\!/\!_2}}$.
\item Prove that the eigen values of real symmetric matrix are real.
\item Express $f(x) = x$ as a half-range cosine series in $0 < x < 2$.

\markA
\partB

\item Test for convergence of the series $\dfrac{1}{1\cdot2\cdot3} +
  \dfrac{3}{2\cdot3\cdot4} + \dfrac{5}{3\cdot4\cdot5} + \ldots\ldots \infty$
\item Discuss the convergence of $\sum\limits_{n = 1}^\infty \dfrac{n^2}{3^n}$.
\item Find the radius of curvature at the point $(a \cos^3\theta, a
  \sin^3\theta)$ on the curve $x^{\!^2\!/_3} + y^{\!^2\!/_3} = a^{\!^2\!/_3}$.
\item Find the minimum value of $x^2 + y^2 + z^2$, when $x + y + z = 3a$.
\item Find the eigen values of A $=
  \begin{bmatrix}
  8 & -6 & 2\\
  -6 & 7 & -4\\
  2 & -4 & 3
  \end{bmatrix}$.
\item Find the Fourier series expansion for $f(x)$, if $f(x) = e^{-x}$ in
  $ 0 < x < 2\pi$.

\markB

\newpage \again

\partCo

\item \iitem Find the Jacobian of $y_1$, $y_2$, $y_3$, with respect to $x_1$,
  $x_2$, $x_3$, if $y_1 = \dfrac{x_2x_3}{x_1}$, $y_2 = \dfrac{x_3x_1}{x_2}$,
  $y_3 = \dfrac{x_1x_2}{x_3}$.
\Or
\item Given the transformation $u = e^x\cos y$ and $v = e^x\sin y$ and that
  $f$ is a function of $u$ and $v$ and also of $x$ and $y$, prove that
  $\dfrac{\partial^2f}{\partial x^2} + \dfrac{\partial^2f}{\partial y^2}
  = (u^2 + v^2) \!\!\left\lgroup \dfrac{\partial^2f}{\partial u^2} +
  \dfrac{\partial^2f}{\partial v^2}\right\rgroup\!\!$.
\ene

\item \iitem State the value of $x$ for which the following series coverges:
  $x - \dfrac{x^2}{2} + \dfrac{x^3}{3} - \dfrac{x^4}{4} + \dfrac{x^5}{5}
  - \ldots\ldots \infty$.
\Or
\item Test the series for convergence:
  \[ 1 + \dfrac{a}{b}x +\dfrac{a(a+1)}{b(b+1)}x^2 +
  \dfrac{a(a+1)(a+2)}{b(b+1)(b+2)}x^3 + \ldots\ldots + \infty,\  (a > 0,
  b > 0, x > 0)\].
\ene

\item \iitem Verify Cayley-Hamilton theorem for the matrix A $ =
  \begin{bmatrix}
  1 & 3 & 7\\
  4 & 2 & 3\\
  1 & 2 & 1
  \end{bmatrix}$\!. Also use it to find A$^{-1}$.
\Or
\item Determine the nature of the following quadratic forms without reducing
  them to canonical forms $x_1^2 + 3x_2^2 + 6x_3^2 + 2x_1x_2 + 2x_2x_3 + 4x_3x_1$.
\ene

\item \iitem Expand $f(x) = x\sin x, 0 < x < 2\pi$, in a Fourier series.
\Or
\item Find the Fourier series expansion for $f(x)$, if $f (x) =
  \begin{cases}
  -\pi, & \! -\pi < x < 0\\
  \ \ x, & \! 0 < x < \pi
  \end{cases}$

  Deduce that $\dfrac{1}{1^2} + \dfrac{1}{3^2} + \dfrac{1}{5^2} + \ldots\ldots
  = \dfrac{\pi^2}{8}$.
\ene
\markC
\ene

\newpage

\mainhead{C 26470}{3}
\comb{APRIL 2012}
\sub{\subject}
\maxtime

\partA

\iitem Give the formula for circle of curvature in Cartesian form.
\item What is Rabbe's test? Define.
\item Find the eigenvalues of the matrix A $= \begin{bmatrix}
  0 & 1 & 1\\
  1 & 0 & 1\\
  1 & 1 & 0
\end{bmatrix}$
\item Test whether $f(x)=|\cos x|$ is odd function or even function in the interval $(-\pi, \pi)$.
\item By using the sine series for $f(x)=1$  in $0<x<\pi$, show that $\dfrac{\pi^2}{8} = 1 + \dfrac{1}{3^2} +
  \dfrac{1}{5^2} + \dfrac{1}{7^2}+\cdots$.

\markA
\partB

\item Discuss the convergence of $\sum\limits_{n=1}^{\infty} \dfrac{(n!)^2}{(2n)!}x^{2n}$.\\
\item If ``$\rho$'' is the radius of curvature at any point $(x,y)$ on the curve $y = \dfrac{ax}{a+x},$
  show that\\
  ${\left( \dfrac{2\rho}{a} \right)}^{\frac{2}{3}} = {\left( \dfrac{x}{y} \right)}^2 + {
     \left(\dfrac{y}{x}\right)}^2$.\\
\item Give the extreme values of the function $f(x,y) = x^3 y^2(12-x-y)$.\\
\item Find the eigenvalues of the matrix A $= \begin{bmatrix}
      1 & 1 & 3\\
      1 & 5 & 1\\
      3 & 1 & 1\\
      \end{bmatrix}$.

\newpage \again

\item Verify Cayley-Hamilton theorem for the matrix A $ = \begin{bmatrix}
       1 & 1 & 1\\
       1 &  2 & -3\\
       2 & -1 & 3 \\
       \end{bmatrix}$.
\item Expand

\begin{tabular}{l l}
  $f(x)$  & $ = \dfrac{1}{4} - x,\ \text{if } 0 < x <\ \dfrac{1}{2}$\\
        &$ = x - \dfrac{3}{4},\ \text{
      if } \dfrac{1}{2} < x < 1$ as the Fourier series of sine terms.
      \end{tabular}\\

\markB
\partCo

\item 
\iitem Find $\dfrac{dx}{\sqrt{1-x^2}} + \dfrac{dy}{\sqrt{1-y^2}} + \dfrac{dz}{\sqrt{1-z^2}}$ 
  if $ x^2 + y^2 + z^2 - 2xyz = 1 $.
\Or

\item Express $\iiint \sqrt{xyz(1-x-y-z)} dx\ dy\ dz $ in terms of $u, v, w$ 
  given that\\      $x + y + z = u,\ y + z = uv,\ z = uvw$.\\
\ene

\item
\iitem Discuss the convergence of the series $x + \dfrac{2^2x^2}{2!} + \dfrac{3^3x^3}{3!} + \dfrac{4^4x^4}{4!} +
  \cdots \cdots \infty$.
\Or
\item Test the series for conditional convergence:

  $\dfrac{1}{2^3} - \dfrac{1}{3^3}(1+2)+\dfrac{1}{4^3}(1+2+3)$
  $ -\dfrac{1}{5^3}(1+2+3+4) + \cdots \cdots + \infty$.\\
\ene

\item 
\iitem Reduce the quadratic form $2x^2_1 + 5x^2_2 + 3x^2_3 + 4x_1x_2$ to canonical form by an 
  orthogonal Transformation.
\Or
\item Find the eigenvalues of A and hence find $\text{A}^n$ (``$n$'' is a positive integer),
  given that \\A $= \begin{bmatrix}
  1 & 2\\
  4 & 3
  \end{bmatrix}$.
\ene

\newpage 

\item
\iitem Find a Fourier series to represent $x - x^2$ from $ x = -\pi$ to $x = \pi$.
\Or
\item Show that for $(-\pi < x < \pi)$,

  \hspace{2cm} $\sin ax = \dfrac{2 \sin (a\pi)}{\pi} \left( \dfrac{\sin x}{1^2-a^2}
   -  \dfrac{2\sin 2x}{2^2-a^2} + \dfrac{3\sin 3x}{3^2-a^2} - \cdots \cdots \right)$.
\ene

\markC
\ene

\newpage

\mainhead{C 15003}{3}
\comb{MAY 2011}
\sub{\subject}
\maxtime

\partA

\iitem Define Convergent and Divergent sequence.
\item Define radius of curvature in Cartesian co-ordinates.
\item The sum of the eigenvalues of a matrix A is equal to the sum ---------
  elements of a given square matrix A.

\item Give the Fourier series for the function $f(x)$ in the interval $\alpha < x <
  \alpha + 2\pi $, giving the definition of Euler's formulae.
\item Given that A $ = \begin{bmatrix}
      5 & 4\\
      1 & 2
\end{bmatrix}$, find its eigenvalue.

\markA
\partB

\item Test for convergence of the series :
  \[ \frac{1}{2\sqrt{1}} + \frac{x^2}{3\sqrt{2}} + \frac{x^4}{4\sqrt{3}} +
  \frac{x^6}{5\sqrt{4}} + \cdots  \infty \].
\item Find the radius of curvature at the point $\left( \dfrac{3a}{2}, \dfrac{3a}{2} \right)$ on the curve
  $x^3 + y^3 = 3axy$.
\item Discuss the convergence of the following series :--
  \[ 1 - \frac{1}{2!} + \frac{1}{4!} + \frac{1}{6!} + \cdots + \infty \].
\item Find the Taylor's series expansion of $e^x \sin y $ near the point
  $\left(-1,\dfrac{\pi}{4}\right)$ upto the third degree terms.

\newpage \again

\item Find the eigenvalues and eigenvectors of the matrix A $ = \begin{bmatrix}
      1 & 1 & 3\\
      1 & 5 & 1\\
      3 & 1 & 1
\end{bmatrix}$.
\item Find a Fourier series to represent $x^2$ in the interval $( -l,\ l )$.

\markB
\partCo

\item
\iitem Verify that the eigenvectors of the real symmetric matrix A $ = \begin{bmatrix}
      3 & -1 & 1\\
      -1 & 5 & -1\\
      1 & -1 & 3 \\
\end{bmatrix}$ are orthogonal in pairs.
\Or
\item  Verify that the eigenvalues of A$^2$ and A$^{-1}$ are respectively the squares and reciprocals of the
       eigenvalues of A, given that A $= \begin{bmatrix}
       3 & 0 & 0\\
       8 & 4 & 0\\
       6 & 2 & 5\\
\end{bmatrix}$.
\ene

\item
\iitem Test for convergence of the series using Raabe's test :
  \[ \sum \frac{4\cdot7 \cdots (3n + 1 )}{1 \cdot 2 \cdots n}x^n. \]
\Or
\item Examine the character of the series :
\iitem  $\sum\limits_{n=1}^{\infty} \dfrac{(-1)^{n-1} n}{2n-1} $ and\\
\item  $\sum\limits_{n=2}^{\infty} \dfrac{(-1)^{n-1} x^n}{n(n-1)} $, $0 < x < 1$.\\
\ene
\ene

\item
\iitem Obtain the Fourier series for

\begin{center}
 \begin{tabular}{l l}
  $f(x)$ & $ = \dfrac{1}{4} - x, \text{ \ if }  0 < x < \dfrac{1}{2}$\\
  & $ = x - \dfrac{3}{4}, \text{ \ if } \dfrac{1}{2} < x < 1.$ 
 \end{tabular}
\end{center}
\Or

\newpage

\item Obtain the first 3 coefficients in the Fourier cosine series for $y$, where $y$ is given in the 
  following table :

\hspace{0.5cm}      \begin{tabular}{c l@{\hspace{0.7cm}} c@{\hspace{0.7cm}}
  c@{\hspace{0.7cm}} c@{\hspace{0.7cm}} c@{\hspace{0.7cm}} c@{\hspace{0.7cm}}c }
      $x$ & :  & 0  & 1  & 2  & 3  & 4  & 5 \\
      $y$ &: &  4 & 8 & 15 & 7 & 6 & 2
\end{tabular}
\ene

\item
\iitem Verify Cayley-Hamilton theorem for the matrix A $ = \begin{bmatrix} 
       1 & 3 & 7 \\
       4 & 2 & 3 \\
       1 & 2 & 1 \\
\end{bmatrix} $ and also find A$^{-1}$.
\Or
\item Diagonalise the matrix A $ = \begin{bmatrix}
      2 & 1 & -1\\
      1 & 1 & -2\\
      -1 & -2 & 1
\end{bmatrix}$ by means of an orthogonal transformation.

\ene

\markC
\ene

\newpage

\def \subject {PTEN/EN09 101---ENGINEERING MATHEMATICS---I}

\mainhead{C 6283}{2}
\comb{MAY 2010}
\sub{\subject}
\maxtime

\partA

\iitem Give the formula for curvature of any given curve in Cartesian form.
\item What is D'Alembert's ratio test?
\item State Cayley-Hamilton Theorem.
\item Find the eigenvalues of $2\text{A}^2$, if A $= \begin{bmatrix} 4 & 1 \\  3 & 2 \end{bmatrix}$.
\item Express $f(x) = x$ as a Fourier series in the interval  $ -\pi < x < \pi$.

\markA
\partB

\item Discuss the convergence of  $\dfrac{5}{2} - \dfrac{7}{4} + \dfrac{9}{6} - \dfrac{11}{8} \cdots$.
\item Find the centre of  curvature of the parabola $y^2 = 12x$ at the point (3, 6).
\item Find the equation of the circle of curvature of the curve $\sqrt{x} + \sqrt{y} =
  \sqrt{a}$ at $\left(\dfrac{a}{4},\dfrac{a}{4}\right)$.

\item Find the eigenvalues of the adjacent matrix A, given
  \[ A = \begin{bmatrix}
    2 & 0 & -1\\
    0 & 2 & 0\\
    -1 & 0 &2
  \end{bmatrix}\]
\item Show that a constant ``C'' can be expanded in an infinite series $\dfrac{4c}{\pi}\left\{
  \sin x + \dfrac{sin 3x}{3} + \dfrac{\sin 5x}{5} + \cdots \right\}$ in the range $0 < x < \pi$.
\item Develop $f(x)$ in Fourier series  in the interval $ (-2, 2)$ if

  \begin{tabular}{l@{} @{}c @{}  r}
    $f(x)$ & $= 0,$ & $-2<x<0$\\
    & $= 1,$ & $0<x<2$
  \end{tabular}.

\markB

\newpage \again 

\partCo

\item \iitem Find the equivalent of

\hspace{1cm}  $\dfrac{\partial^2u}{\partial x^2} + \dfrac{\partial^2u}{\partial y^2}$ in polar co-ordinates.
\Or
\item
  If $ x = r \cos \theta, y = r \sin \theta,$ verify that $\dfrac{\partial(x,y)}{\partial(r,\theta)} \times
 \dfrac{\partial(r,\theta)}{\partial(x,y)} = 1$.
\ene

\item \iitem Test whether the series

  \hspace{1cm} $1 + \dfrac{1}{2^2} - \dfrac{1}{3^2} - \dfrac{1}{4^2} + \dfrac{1}{5^2} +
  \dfrac{1}{6^2} - \dfrac{1}{7^2}
  - \cdots$ is convergent or not.
\Or
\item State the values of $x$ for which the following series converge

\hspace{1cm} $\dfrac{1}{1-x} + \dfrac{1}{2(1-x)^2} + \dfrac{1}{3(1-x)^3} + \cdots + \infty$.
\ene

\item \iitem Reduce the quadratic form $ 2x_1^2 + x_2^2  + x_3^2  + 2x_1x_2  + 2x_1x_3 -4x_2x_3$ to canonical
  form by an orthogonal transformation.
\Or
\item Diagonalise the matrix A = 

  \hspace{1cm} $\begin{bmatrix}
    2 & 1 & -1\\
    1 & 1 & -2\\
    -1 & -2 & 1
    \end{bmatrix}$

  by means of orthogonal transformation.
\ene

\item \iitem Obtain the Fourier series for the function $f(x)$ given by 

  \hspace{1cm} 
  \begin{tabular}{l@{} @{}c @{}  r}
    $f(x)$ & $= 1+\dfrac{2x}{\pi},$ & $-\pi\leq x\leq 0$\\\\
    & $= 1-\dfrac{2x}{\pi},$ & $0<x\leq\pi$
  \end{tabular}
\Or
\item Expand $f(x) = e^{-x}$ as a Fourier series in the interval $(-l,l)$.
\ene

\markC
\ene

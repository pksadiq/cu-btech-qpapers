%
% This file is part of Calicut University Question Paper Collection.
%
% Copyright (c) 2012-2015 Mohammed Sadik P. K. <sadiq (at) sadiqpk (d0t) org>.
% License: GNU GPLv3 or later
%
% Calicut University Question Paper Collection is free software: you can
% redistribute it and/or modify
% it under the terms of the GNU General Public License as published by
% the Free Software Foundation, either version 3 of the License, or
% (at your option) any later version.
% 
% Calicut University Question Paper Collection is distributed in the hope
% that it will be useful,
% but WITHOUT ANY WARRANTY; without even the implied warranty of
% MERCHANTABILITY or FITNESS FOR A PARTICULAR PURPOSE.  See the
% GNU General Public License for more details.
% 
% You should have received a copy of the GNU General Public License
% along with Calicut University Question Paper Collection.
% If not, see <http://www.gnu.org/licenses/>.
% 
%

\def \subj{AI 09 501---ADVANCED MICROPROCESSORS AND MICROCONTROLLERS}

\mainhead{D 30968}{1}
\semfive{OCTOBER 2012}
\sub{\subj}
\maxtime

\partA

\iitem What is the purpose of queue?
\item During what time period(s) of the bus cycle is DEN active?
\item What is register starvation in Pentium?
\item What is the function of RSI and RSO bits of PSW of 8051.
\item Write a program to find the 2's complement of a number using 8051.

\markA
\partB

\item Explain the Buffered bus system of 8086.
\item Write a program to find the maximum value of a 16 bit word.
\item Explain about the MMU of 80386.
\item Write a program to generate a delay of 20 msec using 8051.
\item Write a program to receive the data serially and send it to PO using interrupts.
\item Write a program to subtract a string of 8 bit data indicated by R$_1$ from a string of data
  indicated by R$_0$, The number of data is indicated by R$_2$.

\markB
\partC

\item \iitem Explain the minimum mode of configuration of 8086.
\Or

\newpage \again

\item Write briefly about: (a) assembler directives; (b) memory interfacing.
\ene

\item \iitem Describe briefly about Pentium processor.
\Or
\item Explain in detail the various descriptor tables of 80386.
\ene

\item \iitem Discuss the addressing modes of 8051 with an example.
\Or
\item With the help of a block diagram, explain the working of 8051 microcontroller.
\ene

\item \iitem Describe how a keyboard is interfaced with 8051. Write a program to support it.
\Or
\item Discuss how a speed of a stepper motor is controlled using 8051.
\ene

\markC
\ene

\newpage

\mainhead{D 20917}{1}
\semfive{OCTOBER 2011}
\sub{\subj}
\maxtime

\partA

\iitem What is an overflow?
\item Is {\tt mov ES1 DS} a legitimate instruction or not?
\item What is superscalar technology?
\item How B-register of 8051 is used?
\item Write a program to SWAP the nibbles of the accumulator.

\markA
\partB

\item Explain about Macro and procedures.
\item Explain about Branch prediction logic.
\item Write a program to find the minimum value in a set of 20 numbers.
\item Write a program to toggle only bit P1.5 continuously every 50 ms.
\item Write a program using interrupts to get data from P1 and send it to P2.
\item Write a program to access a byte of data. Which is in data ROM, divide it by
  2 and save the quotient in the data RAM.

\markB
\partC

\item \iitem Explain the working of 8086 with the functional block diagram.
\Or

\newpage \again

\item Explain briefly about 8087 architecture.
\ene

\item \iitem Explain about MMU and superscalar architecture.
\Or
\item Discuss briefly about Pentium.
\ene

\item \iitem Explain the working of 8051.
\Or
\item Discuss about addressing modes of 8051.
\ene

\item \iitem Explain the interface of 8255 PPI with 8051. Write a program
  to support the model operation.
\Or
\item Explain the different modes of timer of 8051.
\ene

\markC
\ene

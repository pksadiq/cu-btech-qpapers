%
% This file is part of Calicut University Question Paper Collection.
%
% Copyright (c) 2012-2015 Mohammed Sadik P. K. <sadiq (at) sadiqpk (d0t) org>.
% License: GNU GPLv3 or later
%
% Calicut University Question Paper Collection is free software: you can
% redistribute it and/or modify
% it under the terms of the GNU General Public License as published by
% the Free Software Foundation, either version 3 of the License, or
% (at your option) any later version.
% 
% Calicut University Question Paper Collection is distributed in the hope
% that it will be useful,
% but WITHOUT ANY WARRANTY; without even the implied warranty of
% MERCHANTABILITY or FITNESS FOR A PARTICULAR PURPOSE.  See the
% GNU General Public License for more details.
% 
% You should have received a copy of the GNU General Public License
% along with Calicut University Question Paper Collection.
% If not, see <http://www.gnu.org/licenses/>.
% 
%

\def \subj{AI 09 502---SIGNALS AND SYSTEMS}

\mainhead{50627}{2}
\semfive{NOVEMBER 2013}
\sub{\subj}
\maxtime

\partA

\iitem Is the signal $x$($t$) $= 10 \cos^2(10\pi t)$ a periodic signal?
\item Define causal system.
\item State sampling theorem.
\item Write down the expression for the complex-exponential fourier series
  expansion of a periodic signal.
\item Find the $z$-transform of $n^2 u(n)$.

\markA
\partB

\item Determine and sketch the even and odd components of the continuous-time
  signal $x(t) = e^{-t} u(t)$.
\item Explain the condition for the stability of a system.
\item Determine and sketch the spectrum of $x(t) = 10\sin 2\pi f_0 t$.
\item Find the fourier transform of $x(t) = 5 \text{e}^{-2|t|}$. Plot its magnitude
  and phase spectra.
\item Given $F(s) = \dfrac{s + 8}{s^2 + 6s + 13}$, find $f(0)$, and $f'(0)$ using
  the initial value theorem.
\item Explain any two properties of z-transform.

\markB
\partC

\item \iitem With suitable examples explain the basic operations on signals.
\Or

\newpage \again

\item An LTI system is described by the differential equation
  $\dfrac{\text{d}y(t)}{\text{d}t} + 6 y(t) = x(t)$. Determine its
  impulse response and sketch its magnitude and phase response.
\ene

\item \iitem Discuss in detail about Hilbert transform and its properties.
\Or
\item Explain the condition for ditortionless transmission and its properties.
\ene

\item \iitem Explain DFT and its properties. Find the DFT of the sequence
  $x(n) = \{-1, 1, -1, 1\}$.
\Or
\item For the DT system described by the following difference equation,
  determine (i) the unit sample response sequence, $h(n)$ (ii) the step
  response sequence $g(n)$ and (iii) whether it is BIBO stable.
  \[ \text{Y} (n) = 0.6y(n - 1) - 0.08y(n - 2) + x(n) \]
\ene

\item \iitem Determine the z-transform and the ROC of the two sided signal
  (i) $x(n) = (0.5)^{|n|}$.\\ (ii) $x(n) = (2)^{|n|}$.
\Or
\item With a suitable example, explain the determination of frequency
  response from poles and zeroes
\ene

\markC
\ene

\newpage

\mainhead{30969}{2}
\semfive{NOVEMBER 2012}
\sub{\subj}
\maxtime

\partA

\iitem Define a continues time signal.
\item State frequency shift theorem of DTFT.
\item What is the Laplace transform of $\delta (t)$? What is its region of convergence?
\item Write down the complex-exponential Fourier series expansion of the periodic
  signal\\ $x(t)=2\cos 2\pi t$.
\item Give an N-length causal sequence, $x(n)$, explain how its DTFT and N-point DFT
  are related.

\markA
\partB

\item Explain linearity and causality with suitable examples.
\item State and prove convolution theorem.
\item What is an Hilbert transform? Explain any two of its properties.
\item What are poles and zeros? How they are useful in stability analysis?
\item Find the Laplace transform of $x(t) = e^{-4t} \sin (2\pi 100t) u(t)$. 
\item Find the unilateral Z-transform of $nu(n)$.

\markB
\partC

\item \iitem Are the following discrete-time signals periodic? If they are,
  what are their fundamental periods?
\iitem $x(n) = 2\cos \left(\dfrac{\pi}{3}\right)n + 3 \sin
  \left(\dfrac{\pi}{4}\right)n$

\newpage \again

\item $x(n) = 2\cos^2 \left(\dfrac{\pi}{6}\right)n $
\ene
\Or
\item \iitem Determine and sketch the even and odd components of the continues-time
  signal $x(t)=e^{-t}u(t).$ 
\item Prove the following: The product of two even signals is even.
\ene
\ene

\item \iitem $X(t) = \dfrac{1}{5} \text{sinc}\,10t.$ Sketch the Fourier transform
  of $x(t).$ If $x(t)$ is sampled at
\iitem $f_s = 10\ sps$
\item $f_s = 5\ sps$
\item $f_s = 2\ sps$.

  Sketch the FT of the sampled signal for each of the above sampling frequencies. 
\ene
\Or
\item 
\iitem Find $x(n)$ if $X(e^{j\omega}) = 4 \cos^2 \omega$
\item Compute the DTFT of $x(n) = [a^n \cos^n \omega_0n]u(n)$; $|a| < 1$.
\ene
\ene

\item \iitem Using Laplace Transform method, solve the following differential
  equations for the initial conditions:

  $\left(\dfrac{\ud^2x(t)}{\ud t^2}\right) + \left( \dfrac{5 \ud x(t)}{\ud t}\right) +
  6x(t) = \delta (t) + 6 u(t)\;$ 
  with $x(0^-) = 1$ and $x'(0^-) = 2$.
\Or
\item \iitem Compare continues Fourier transform and discrete-time Fourier transform.
\item Find the circular convolution of the two causal sequences
  $\{X(n)\} = \{1,2,3,4\}$ and $ y(n) = \{ 4,3,2,1\}$.
\ene \ene

\item \iitem Determine the Z-transform and the ROC of the two-sided signals 
\iitem $x(n) = (0.5)^{|n|}$ \item $x(n) = (2)^{|n|}$
\ene
\Or
\item Discuss in detail about the analysis of LTI systems. 
\ene

\markC
\ene

\newpage

\mainhead{20918}{2}
\semfive{NOVEMBER 2011}
\sub{\subj}

\maxtime
\partA

\iitem State the difference between energy and power signals.
\item Find the area under the signal $x(t) = 10\delta(t - 2)$.
\item Find the Fourier transform of $x(t) = 1/t.$
\item What is the significance of power spectral density?
\item State the initial value theorem.

\markA
\partB

\item Determine and sketch the even and odd components of the continuous-time
  signal $x(t) = e^{-t} u(t)$.
\item State and prove convolution theorem.
\item State and explain Drichlet's condition for the convergence of Fourier series.
\item Determine the DTFT of $y_1(n) = x(2n)$, given that DTFT of $x[n] = X (e^{j\omega})$.
\item Find the Laplace transform of $x(t) = e^{-3t} \cos (2\pi 100 t) u(t)$.
\item Find the unilateral Z-transform of $n^2 u(n)$.

\markB
\partCo

\item \iitem Explain the classification of signals with suitable examples.
\Or

\newpage \again

\item \iitem What is a Linear Time Invariant System? Explain.
\item A particular LTI system has $h(t) = e^{-2t} u(t)$. Determine its output 
  signal $y(t)$ corresponding to an input signal $x(t) = u(t)$.
\ene \ene

\item \iitem State and prove Parseval's theorem.
\Or
\item If $x(t) = \left[ \dfrac{t^{n - 1}}{(n-1)!}\right] e^{-at} u(t)$, where $a > 0$.
  Show that $X(f) = \dfrac{1}{(a + j\omega)^n}.$
\ene

\item \iitem Find the circular convolution of the two causal sequences $\{x(n)\} =
  \{1, 2, 3, 4\}$ and $y(n) = \{4, 3, 2, 1\}$ by using DFT and IDFT.
\Or
\item Using Laplace transform method, solve the following differential equation for 
  the initial conditions.
\[ \dfrac{\ud^2 x(t)}{\ud t^2} + \dfrac{5\ud x(t)}{\ud t} + 6x(t) = \delta(t) + 6u(t)\]

with $x(0^-) = 1$ and $x'(0^-) = 2$.
\ene

\item Find the unilateral Z-transform of $x(n) = [ a^n \cos \omega_0 n] u(n).$
\Or 
\item For the Discrete time system described by the following difference equation,
  determine 
\iitem the unit sample response sequence $h(n)$
\item the step response sequence $g(n)$ and
\item whether it is BIBO stable

\[ y(n) = 0.6 y(n - 1) - 0.08 y(n - 2) + x(n)\]
\ene 

\markC
\ene 

%
% This file is part of Calicut University Question Paper Collection.
%
% Copyright (c) 2012-2015 Mohammed Sadik P. K. <sadiq (at) sadiqpk (d0t) org>.
% License: GNU GPLv3 or later
%
% Calicut University Question Paper Collection is free software: you can
% redistribute it and/or modify
% it under the terms of the GNU General Public License as published by
% the Free Software Foundation, either version 3 of the License, or
% (at your option) any later version.
% 
% Calicut University Question Paper Collection is distributed in the hope
% that it will be useful,
% but WITHOUT ANY WARRANTY; without even the implied warranty of
% MERCHANTABILITY or FITNESS FOR A PARTICULAR PURPOSE.  See the
% GNU General Public License for more details.
% 
% You should have received a copy of the GNU General Public License
% along with Calicut University Question Paper Collection.
% If not, see <http://www.gnu.org/licenses/>.
% 
%
\def \subj{AI09 304---ELECTRICAL ENGINEERING}

\mainhead{D 51037}{2}
\semthree{NOVEMBER 2013}
\sub{\subj}
\maxtime

\partA

\iitem What are the desirable conditions for DC generators to be
  connected in parallel?
\item Why is the HV side of a single-phase transformer left open and
  the open circuit test is performed by energising the LV side?
\item Why is a single-phase transformer rated in kVA?
\item Give the relationship between the gross mechanical power output
  and the air gap power in a three-phase Induction motor.
\item What is meant by creep in an Induction type energy meter?

\markA
\partB

\item Classify the DC generators and draw the external characteristics
  of each type of DC generator.
\item Explain the principle of operation of a single-phase transformer and
  hence derive and expression for induced \emph{emf} from its first principles.
\item Discuss the phenomenon of armature reaction in a three-phase alternator
  for pure inductive load at the armature terminals.
\item Derive the condition for maximum torque at running condition for a
  three-phase induction motor.
\item Explain the possible errors that can occur in a moving iron instrument
  when both AC and DC quantities are measured.
\item Demonstrate how power can be measured in a three-phase circuit by using
  two wattmeters.

\markB

\newpage \again

\partCo

\item \iitem \iitem Differentiate Lap and wave windings in a DC machine. \marko{2}
\item Two DC shunt generators are rated 230 kW and 150 kW, 400 V. Their
  full load voltage drops are 3\% and 6\% respectively. They are excited
  to no-load voltages of 410 V and 420 V respectively. How will they share
  a load of 1000 Amps and the corresponding bus voltage? \marko{8}
\ene
\Or
\item \iitem Compare the advantages of four point starter over three-point
  starter. \marko {2}
\item A 10 kW, 250 V shunt motor has an armature resistance of 0.5$\Omega$ and
  a field resistance of 200 W. AT no-load and rated voltage, the speed is 1200
  \emph{rpm}, and the armature current is 3 Amps. At full load 8 rated voltage,
  the line current is 47 Amps and because of armature reaction, the flux is 4\%
  less than its no-load value. What is the full-load speed? What is the developed
  torque at full load? \marko{8}
  \ene
\ene

\item \iitem Explain the different methods of cooling adopted in a single-phase
  transformer. \marko{10}
\Or
\item \iitem What is all day efficiency? \marko{2}
\item Draw and explain the phasor diagram respecting the relationship between
  different voltages and current of a single-phase transformer if the secondary
  is loaded with pure resistive and pure capacitive load. \marko{8}
  \ene
\ene

\item \iitem Explain in detail the construction of a three-phase Induction motor.
  Also explain the different types of rotor construction. \marko {10}
\Or
\item Explain how does a synchronous motor perform when the excitation current is
  increased from zero to rated value with the help of phasor diagram relating the
  induced voltage and the field currents. \marko{10}
\ene

\item \iitem Explain the kelvins double bridge method of measurement of Low
  resistance. \marko{10}
\Or
\item Explain how dissipation factor can be determined accurately with the help
  of schering Bridge. \marko{10}
\ene

\markC
\ene

\newpage

\mainhead{D 20630}{3}
\semthree{OCTOBER 2011}
\sub{\subj}
\maxtime

\partA

\iitem What is the need for starters for starters used in D.C. motors? Justify your answer.
\item What is the significance of `Back emf' in a shunt motor?
\item Mention the losses that occur in a single phase transformer.
\item Comment on the starting torque of a wound rotor induction motor.
\item Which bridge is suitable for measuring small capacitance? Mention any {\em two} special
  features of it.

\markA
\partB

\item A 75kW, 250 V compound dc generator has the following data:

\hspace{1cm}
\begin{tabular}{ l l }
  R$_\text{a} = 0.04 \Omega$ & R$_\text{se} = 0.004 \Omega$\\
  R$_\text{sh} = 100 \Omega$ & Brush contact deep = 1V per brush. \\
\end{tabular}

Determine and compare the induced \emph{emf} in the generator when it is

\iitem Long shunt compound and
\item Short shunt compound.
\ene
\item Draw the phasor diagram representing the various currents and voltage in a single-phase transformer
  when loaded with inductive and capacitive loads.
\item Discuss how constant flux is maintained in a single-phase transformer when the secondary winding is
  loaded.
\item Explain how the rotor rotates in a three-phase induction motor when a 3-$\phi$ supply is given to
  the stator terminals.
\item Explain any {\em one} starting method for a three-phase synchronous motor.

\newpage \again

\item Explain the constructional details of a permanent magnet moving coil type ammeter.

\markB
\partCo

\item \iitem Explain with appropriate graphs the magnetisation and load characteristics of a dc shunt 
  generator (self excited) type.
\Or
\item A dc shunt motor rated 10kW connected to 250 V supply is loaded to draw 35 A armature current 
  running at 1250 rpm. Given R$_\text{a} = 0.5 \Omega$. Determine the following:
\iitem Load torque if rotational loss is 500$ \Omega$.
\item Motor efficiency if R$_\text{sh} = 250 \Omega$.
\item Armature current for maximum motor efficiency and the maximum efficiency.
\ene

What is the corresponding load torque speed? \marko{10}
\ene

\item \iitem Explain constructional details of shell and core type transformers.
\Or
\item \iitem Derive the emf equation of a single-phase transformer. \marko{5}
\item Draw the equivalent circuit of a single-phase transformer and explain. \marko{5}
\ene\ene

\item \iitem Explain with neat diagram the construction and principle of working of a three synchronous
  motor.
\Or
\item \iitem Draw the equivalent circuit of a three-phase induction motor and explain the circuit parameters.
\item A 6 pole 50 Hz 3-phase induction motor running on a full load develops a useful torque
  of 160 Nm when the rotor emf makes 120 complete cycles per minute. Calculate the shaft power output.
  If the mechanical torque lost in friction and that for core-loss is 10 Nm, calculate the
\iitem Copper loss in the rotor windings.
\item The input to the motor and
\item Efficiency.
\ene

The total stator loss is given to be 800 W.
\ene\ene

\item \iitem Justify that two wattmeters are required to measure three-phase power, with appropriate equations.

\item Explain with suitable diagrams how inductance can be measured in a Maxwell's bridge in comparison with a 
  standard variable capacitance. Also discuss the advantage and disadvantages of the Maxwell's bridge.
\ene

\markC
\ene
